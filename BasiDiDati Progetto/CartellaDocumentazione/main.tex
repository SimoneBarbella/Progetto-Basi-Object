\documentclass[a4paper,12pt]{article}

% Pacchetti di base
\usepackage[utf8]{inputenc}
\usepackage[T1]{fontenc}

\usepackage{makecell}

\usepackage{graphicx} % Per immagini
\usepackage{pdfpages}

\usepackage{geometry}
\usepackage{longtable}
\usepackage[table]{xcolor}
\usepackage{titlesec}
\usepackage{array}

\usepackage{longtable}
% da non utlizzare, va in conflitto con graphihicx \usepackage[margins=1in]{geometry} % Margini ottimizzati
\usepackage{multirow, tabularx}
\usepackage{tcolorbox}
\usepackage{hyperref}

%




\usepackage{colortbl}   % Per colorare celle della tabella
\usepackage{float}      % Per controllare la posizione delle tabelle



% Configurazioni per link e titoli
\hypersetup{
    colorlinks=true,
    linkcolor=black,
    urlcolor=black
}
\renewcommand{\contentsname}{Indice}
\renewcommand{\listtablename}{Elenco delle tabelle}


\begin{document}


\begin{figure} % [h] = here, posizione suggerita
    \centering
    \includegraphics[width=0.5\textwidth]{CartellaDocumentazione/Unina.png}

\end{figure}




% Titolo migliorato
\title{UninaFoodLab}








\author{
    Nome e cognome \\ Matricola
    \and
    Simone Barbella \\ N86004906
}
\date{Anno Accademico 2024/2025}



% Creazione del titolo
\maketitle



\clearpage

% Indice e lista delle tabelle
\tableofcontents

\section{Presentazione del Progetto}
\subsection{Cos'è UninaFoodLab}
UninaFoodLab nasce per modernizzare l'insegnamento culinario, unendo la flessibilità dell'apprendimento digitale con l'imprescindibile esperienza pratica in laboratorio.
L'obiettivo principale non è solo la gestione didattica, ma l'ottimizzazione delle risorse: il sistema è progettato per calcolare con precisione i fabbisogni di materie prime, riducendo drasticamente lo spreco alimentare (Food Waste) attraverso un meccanismo di conferme puntuali.
 
\section{Progettazione Concettuale}
\subsection{Diagramma ER}

\begin{figure}[!h] % [h] = here, posizione suggerita
    \centering
    \includegraphics[width=0.8\textwidth]{CartellaDocumentazione/DiagrammaER1.pdf}
\end{figure}

\subsubsection{Descrizione Diagramma ER}
qua si scrive qualcosa

\subsection{Diagramma UML}
Irealtà va quello UML e non Quello ER, ma non è finito 
\begin{figure}[!h] % [h] = here, posizione suggerita
    \centering
    \includegraphics[width=0.8\textwidth]{CartellaDocumentazione/DiagrammaER1.pdf}
\end{figure}

\subsubsection{Descrizione Diagramma UML}

qua si scrive 

\subsection{Ristrutturazione Diagramma UML}
Irealtà va quello UML e non Quello ER, ma non è finito 
\begin{figure}[!h] % [h] = here, posizione suggerita
    \centering
    \includegraphics[width=0.8\textwidth]{CartellaDocumentazione/DiagrammaER1.pdf}
\end{figure}

\subsubsection{Descrizione Delle Scelte}
\subsubsection{Descrizione Accorpamento Entità e Assocazini}
\subsubsection{Descrizione Eliminazioni delle gerarchie }
\subsection{Dizionari}
\subsubsection{Dizionario delle Classi}

\begin{table}[H] % oppure [htbp]
\label{tab:classi1}

\centering
\resizebox{0.9\textwidth}{!}{
\begin{tabular}{|>{\centering\arraybackslash}m{0.22\textwidth}|>
                {\centering\arraybackslash}m{0.28\textwidth}|>{\raggedright\arraybackslash}m{0.45\textwidth}|}

                \hline
    \textbf{Classe} & \textbf{Descrizione} & \textbf{Attributi} \\
    \hline
    \multirow{12}{=}{\centering Utente} & \multirow{12}{=}{\centering Descrittore di ogni Studente o Chef } &
\texttt{email} [PK](String): è un indirizzo email unico e valido che permette di contattare l’utente.\\

& & \texttt{nome} (String): è il nome dell'utente.\\

& & \texttt{cognome} (String): è cognome dell'utente. \\

& & \texttt{password} (String): una password permette all'utente di proteggere il suo account e le sue informazioni.\\

& & \texttt{matricola} (String): string alfanumerica che identifica lo studente. \\ 

& & \texttt{tipoUtente} (TipoUtente): indica il tipo di utente: (chef), (studente) oppure (chefStudente). \\


\hline
\multirow{8}{=}{\centering Corso} & \multirow{8}{=}{\centering Descrittore di ciascun corso creato} &

\texttt{id\_corso} [PK] (Int): identificativo univoco di Corso.\\

&& \texttt{dataInizio} (Data): indica l'inizio del corso.\\

& & \texttt{nome} (String): è il nome assegnato al corso. \\

& & \texttt{frequenza} (String): indica ogni quanti giorni il corso si svolge.\\

& & \texttt{numPartecipanti} (Int): Numero dei partecipanti al corso. \\
& & \texttt{numSessioni} (Int): specifica quante sessioni ha un corso. \\
\hline

\multirow{3}{=}{\centering Iscrizione} & \multirow{3}{=}{\centering Descrittore dell'iscrizione degli utenti} &
\texttt{dataIscrizione} (Data): indica l'inizio dell'iscrizione per un corso.\\

& & \texttt{Stato} (String): indica lo stato dell'iscrizione di uno studente. \\
\hline

\end{tabular}
}

\end{table}




\begin{table}[H] % oppure [htbp]
\label{tab:classi2}
\centering
\resizebox{0.9\textwidth}{!}{
\begin{tabular}{|>{\centering\arraybackslash}m{0.22\textwidth}|>
                {\centering\arraybackslash}m{0.28\textwidth}|>{\raggedright\arraybackslash}m{0.45\textwidth}|}

                \hline
    \textbf{Classe} & \textbf{Descrizione} & \textbf{Attributi} \\
    \hline
    \multirow{12}{=}{\centering Sessione} & \multirow{12}{=}{\centering Descrittore di ciascuna sessione presente in un corso} & 
    \texttt{id\_sessione} [PK] (Int): identificativo univoco di Sessione\\
    &&\texttt{oraInizio} (Data): orario di inizio della sessione.\\

& & \texttt{numAderenti} (Int): numero di partecipanti alla sessione. \\

& & \texttt{teoria} (String): argomento teorico spiegato in sessione.\\

& & \texttt{tipoSessione} (TipoSessione): indica il tipo di sessione: (online) oppure (presenza). \\ 



\hline
\multirow{6}{=}{\centering Ricetta} & \multirow{6}{=}{\centering Descrittore di ciascuna ricetta usata in una sessione} &
\texttt{id\_ricetta} [PK] (Int): identificativo univoco di Ricetta.\\

&& \texttt{nome} (String): nome della ricetta.\\

& & \texttt{descrizione} (String): la preaparazione della ricetta. \\

& & \texttt{tempo} (Time): tempo per la preparazione della ricetta. \\


\hline

\multirow{4}{=}{\centering Ingrediente} & \multirow{4}{=}{\centering Descrittore di ciascun ingrediente usato per una ricetta} &
\texttt{nome} [PK] (String): nome dell'ingrediente.\\

& & \texttt{unitàDiMisura} (String): specifica dell'unità di misura di un ingrediente ES: Kg, litri ecc.. .\\

\hline


\centering Adesione & \centering Classe associativa per gestire l'associazione tra Utente (studente/chefStudente) e Sessione (presenza) &
\texttt{dataAdesione} (Data): data relativa all'adesione di uno studente ad una sessione.\\


\hline

\centering Richiede & \centering Classe associativa per gestire l'associazione tra Ingrediente e Ricetta &
\texttt{quantitàNecessaria} (Decimal): quantità utilizzata di ogni singolo ingrediente per una ricetta.\\

\hline

\centering Categoria\_Corso & \centering Classe per gestire l'attributo multivalore categoria del corso &
{\ttfamily categoria} (String): una categoria associata a un corso.\\

\hline

\centering Specializzazione\_ \space Chef & \centering Classe per gestire l'attributo multivalore specializzazione dello chef. &
{\ttfamily specializzazione} (String): una specializzazione associata a uno chef.\\

\hline

\end{tabular}
}

\end{table}

\subsubsection{Dizionario Associazioni}


\begin{table}[H] % oppure [htbp]
\label{tab:classi} 
\centering
%\resizebox{1.\textwidth}{!}{
\begin{tabular}{|>{\centering\arraybackslash}m{0.22\textwidth}|>
                {\centering\arraybackslash}m{0.28\textwidth}|>{\raggedright\arraybackslash}m{0.45\textwidth}|}
                \hline
    \textbf{Associazione} & \textbf{Descrizione} & \textbf{Molteplicità delle Classi coinvolte} \\
    \hline
  \centering richiede (adesione) & \centering Richiesta di adesione da parte di uno studente & Utente [0..*] - Adesione [1] \\



\hline

  \centering richiede (iscrizione) & \centering Richiesta di iscrizione da parte di uno studente & Utente [0..*] - Iscrizione [1] \\

\hline

  \centering gestisce & \centering Gestione di un corso da parte di uno chef & Utente [0..*] - Corso [1..*] \\
\hline

  \centering compone & \centering Un corso è composto da più sessioni  & Corso [1] - Sessione [1..*] \\
\hline  



  \centering prepara & \centering Ad ogni sessione in presenza si prepara una ricetta & Sessione [1..*] - Ricetta [1..*] \\

\hline

  \centering haBisogno & \centering ogni ricetta ha bisogno di sapere la quantità degli ingredienti  & Ricetta [1] - Richiede [1..*] \\

\hline


  \centering necessita & \centering Un ingrediente necessita di sapere la quantità per la ricetta & Richiede [1..*] - Ingrediente [1] \\

\hline

  \centering per & \centering Richiesta di adesione per una sessione in presenza & Adesione [1] - Sessione [0..*] \\

\hline

  \centering relativa & \centering Iscrizione di uno studente per un corso & Iscrizione [1] - Corso [1..*] \\

\hline 
   \centering specializzatoIn & \centering Specializzazione di uno chef & Utente [1] - Specializzazione\_Chef [0..*] \\

\hline
   \centering categorizzatoIn & \centering Categoria a cui appartiene un corso & Corso [1] - Categoria\_Corso [1..*] \\

\hline

\end{tabular}
%}

\end{table}
\subsubsection{Dizionario delle Viste}

\begin{table}[H] % oppure [htbp]
\label{tab:classi} 
\centering
%\resizebox{1.\textwidth}{!}{
\begin{tabular}{|>{\centering\arraybackslash}m{0.22\textwidth}|>
                {\centering\arraybackslash}m{0.28\textwidth}|>{\raggedright\arraybackslash}m{0.45\textwidth}|}
                \hline
    \textbf{Viste} & \textbf{Descrizione} & \textbf{Attributi} \\
    \hline
  \centering Vista\_Fabbisogni\_\\Sessione & \centering Struttura virtuale che aggrega i dati logistici. Incrocia le adesioni e le ricette per calcolare il fabbisogno puntuale di materie prime & quantità\_totale: (Derivato) Valore numerico che esprime il fabbisogno complessivo dello specifico ingrediente per la specifica sessione (calcolato come: Dose Unitaria × Numero Partecipanti). \\
  \hline

\end{tabular}
%}
\end{table}
\subsubsection{Dizionario Vincoli}


\renewcommand{\arraystretch}{1.3}

\begin{center}
\begin{longtable}{|>{\centering\arraybackslash}m{0.30\textwidth}|
                  >{\centering\arraybackslash}m{0.22\textwidth}|
                  >{\raggedright\arraybackslash}m{0.45\textwidth}|}
\caption{Dizionario dei vincoli} \label{tab:vincoli} \\

\hline
\textbf{Vincolo} & \textbf{Tipo} & \textbf{Descrizione} \\
\hline
\endfirsthead

\hline
\textbf{Vincolo} & \textbf{Tipo} & \textbf{Descrizione} \\
\hline
\endhead

Email & Dominio & L’attributo “email” della classe Utente deve contenere una combinazione non nulla di lettere, numeri e simboli, seguiti da una chiocciola (“@”), altre lettere/simboli, un punto (“.”), e finire con almeno due lettere. \\
\hline

Password & Dominio & L’attributo “password” della classe Utente deve essere lungo almeno 8 caratteri, e avere almeno una lettera maiuscola, una minuscola, un numero e un carattere speciale. \\
\hline

Nome & Dominio & L’attributo “Nome” della classe Utente deve contenere solo stringhe di caratteri alfabetici, non vuote e con lunghezza <= 50 caratteri. \\
\hline

Cognome & Dominio & L’attributo “Cognome” della classe Utente deve contenere solo stringhe di caratteri alfabetici, non vuote e con lunghezza <= 50 caratteri. \\
\hline

TipoUtente & Dominio & L’attributo "tipoUtente" della classe Utente è un’enumerazione che può assumere solo i valori: chef, studente, chefStudente, per identificare il ruolo dell’utente nel sistema. \\
\hline

Specializzazione & N‑tupla & Se l’attributo “TipoUtente” è chef o chefStudente, allora l’attributo “Specializzazione” assume un valore alfabetico con lunghezza <= 50. \\
\hline

Matricola & N‑tupla & Se l’attributo “TipoUtente” è studente o chefStudente, allora l’attributo “Matricola” assume un valore alfanumerico con lunghezza <= 50. \\
\hline

DataInizio & Dominio & L'attributo "DataInizio" della classe Corso deve essere una data valida nel formato previsto, non può essere nulla. \\
\hline

Nome & Dominio & L'attributo "Nome" della classe Corso deve contenere una stringa non vuota con solo caratteri alfabetici e lunghezza massima 50 caratteri. \\
\hline

Categoria & N‑tupla & L'attributo "Categoria" della classe Corso deve contenere almeno una categoria. Ogni elemento deve essere una stringa non vuota. \\
\hline

Frequenza & Dominio & L'attributo "Frequenza" della classe Corso deve contenere una stringa che rappresenta un intervallo temporale coerente. \\
\hline

NumPartecipanti & Intrarelazionale & L'attributo "NumPartecipanti" della classe Corso è calcolato automaticamente in base al numero di iscrizioni associate. \\
\hline

NumSessioni & Dominio & L'attributo "NumSessioni" della classe Sessione deve essere un intero positivo >0. \\
\hline

OraInizio & Dominio & L'attributo "OraInizio" della classe Sessione deve essere una data valida, non nulla. \\
\hline

TipoSessione & Dominio & L'attributo "TipoSessione" della classe Sessione è un’enumerazione che può assumere solo i valori: presenza, online. \\
\hline

Quantità & Intrarelazionale & Se l'attributo "TipoSessione" è "presenza", allora "Quantità" assume un valore numerico >0. \\
\hline

NumAderenti & Intrarelazionale & Se l'attributo "TipoSessione" è "presenza", allora "NumAderenti" assume un valore numerico valido >0. \\
\hline

Teoria & Intrarelazionale & Se l'attributo "TipoSessione" è "online", allora "Teoria" deve essere una stringa alfanumerica con lunghezza <=255. \\
\hline

Nome & Dominio & L'attributo "Nome" della classe Ricetta deve contenere una stringa alfabetica non vuota con lunghezza <=50. \\
\hline

Descrizione & Dominio & L'attributo "Descrizione" della classe Ricetta deve contenere una stringa alfabetica non vuota con lunghezza <=50. \\
\hline

Tempo & Dominio & L'attributo "Tempo" della classe Ricetta deve contenere una data (ora, minuti, secondi) valida e non vuota. \\
\hline

Nome & Dominio & L'attributo "Nome" della classe Ingrediente deve contenere una stringa alfabetica non vuota con lunghezza <=50. \\
\hline

UnitàDiMisura & Dominio & L'attributo "UnitàDiMisura" della classe Ingrediente deve contenere una stringa alfabetica non vuota con lunghezza <=50. \\
\hline

\end{longtable}
\end{center}




\section{Schema Logico}
\subsection{Tabelle}
Leggenda:
\begin{itemize}
    \item Le parole con il \textbf{grassetto} sono indicate tutti i nomi delle tabelle
    \item Le parole \underline {sottolinete} una volta indicano le chiavi primarie
    \item Le parole \underline{\underline{sottolinate}} due volte indicano le chiavi esterne
    \item Le parole in \textit{italico} indicano che una chiava primaria è composta da piu attributi 
\end{itemize}

\vspace{2em}

\renewcommand{\arraystretch}{1.80}
\begin{tabular}{lp{0.7\textwidth}} \hline
    \textbf{Utente} & email, nome, cognome, password, specializzazione, matricola, tipoUtente, id\_corso\\ 
    \hline
    \textbf{Corso} & id\_corso, dataInizio, nome, categoria, frequenza, numPartecipanti, id\_sessione \\
    \hline
    \textbf{Sessione} & id\_sessione, numSessioni, oraInizio, numAderenti, quantità, teoria, TipoSessione, id\_ricetta, id\_corso\\
    \hline
    \textbf{Ricetta} & id\_ricetta, nome, descrizione, tempo, id\_sessione \\
    \hline
    \textbf{Ingrediente} & nome, unitàDiMisura\\
    \hline
    \textbf{Adesione} & dataAdesione, email,id\_sessione\\
    \hline
    \textbf{Iscrizione} & dataIscrizione, stato, email,id\_corso\\
    \hline
    \textbf{Richiede} & quantitàNecessaria, id\_ricetta, nomeIngrediente \\
    \hline
\end{tabular}



\listoftables

\end{document}

