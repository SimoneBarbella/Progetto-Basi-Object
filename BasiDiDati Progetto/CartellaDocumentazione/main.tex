\documentclass[a4paper,12pt]{article}

% Pacchetti di base
\usepackage[utf8]{inputenc}
\usepackage[T1]{fontenc}

\usepackage{makecell}

\usepackage{graphicx} % Per immagini
\usepackage{pdfpages}

\usepackage{geometry}
\usepackage{longtable}
\usepackage[table]{xcolor}
\usepackage{titlesec}
\usepackage{array}
\usepackage[dvipsnames]{xcolor} % 'dvipsnames' offre più nomi di colori

\usepackage{longtable}
% da non utlizzare, va in conflitto con graphihicx \usepackage[margins=1in]{geometry} % Margini ottimizzati
\usepackage{multirow, tabularx}
\usepackage{tcolorbox}
\usepackage{hyperref}

%




\usepackage{colortbl}   % Per colorare celle della tabella
\usepackage{float}      % Per controllare la posizione delle tabelle



% Configurazioni per link e titoli
\hypersetup{
    colorlinks=true,
    linkcolor=black,
    urlcolor=black
}
\renewcommand{\contentsname}{Indice}
\renewcommand{\listtablename}{Elenco delle tabelle}


\begin{document}


\begin{figure} % [h] = here, posizione suggerita
    \centering
    \includegraphics[width=0.5\textwidth]{CartellaDocumentazione/immagini/Unina.png}
\end{figure}




% Titolo migliorato
\title{UninaFoodLab}








\author{
    Antonino De Martino \\ N86005103 
    \and 
    \\ \\ Gruppo \\ OOBD57
    \and
    Simone Barbella \\ N86004906
}

\date{Anno Accademico 2024/2025}



% Creazione del titolo
\maketitle



\clearpage

% Indice e lista delle tabelle
\tableofcontents

\section{Presentazione del Progetto}
\subsection{Cos'è UninaFoodLab}
Con UninaFoodLab, la didattica culinaria entra in una nuova era digitale, senza però perdere il contatto con la materia prima. Il progetto fonde la comodità dello studio teorico online con la necessità fondamentale della pratica in cucina, creando un ecosistema formativo completo e moderno.
\\
L'aspetto più rivoluzionario del progetto risiede nella sua forte vocazione alla sostenibilità. L'obiettivo primario non si limita alla gestione di orari e corsi, ma punta dritto all'efficienza logistica: eliminare gli sprechi. Grazie a un sistema avanzato di adesioni e calcolo dei fabbisogni, UninaFoodLab trasforma la gestione della dispensa: non si acquista più in base a stime approssimative, ma in base alle reali necessità confermate. Questo approccio permette di ridurre drasticamente lo spreco alimentare, promuovendo una cultura del cibo più etica e consapevole all'interno dell'ambiente accademico.
 
\section{Progettazione Concettuale}
\subsection{Diagramma ER}

\begin{figure}[!h] % [h] = here, posizione suggerita
    \centering
    \includegraphics[width=0.8\textwidth]{CartellaDocumentazione/diagrammi/DiagrammaER1.pdf}
\end{figure}

\subsubsection{Descrizione Diagramma ER}
Il diagramma Entità-Relazione (ER) proposto definisce l'architettura informativa della piattaforma UninFoodLab, progettata per digitalizzare i processi di un laboratorio culinario accademico. Il modello garantisce l'integrità referenziale e supporta le regole di business attraverso una struttura modulare.\\
La descrizione del diagramma è articolata nelle seguenti sei aree tematiche, che riflettono la logica sequenziale dei processi gestiti:
\begin{enumerate}
    \item \textbf{Gestione Utenti e Gerarchie}: Definisce gli attori del sistema, implementando la specializzazione dei ruoli tra corpo docente (Chef) e discenti (Studenti);
    \item \textbf{Corsi e Iscrizioni}: Modella l'offerta formativa e le relazioni di titolarità e fruizione, gestendo il ciclo di vita delle iscrizioni;
    \item \textbf{Struttura delle Sessioni}: Dettaglia la scomposizione temporale dei corsi, introducendo la distinzione strutturale tra didattica online (teorica) e in presenza (pratica);
    \item \textbf{Logistica in Presenza}: Approfondisce le dinamiche operative del laboratorio fisico, regolando i flussi di prenotazione degli studenti (Adesioni) e il calcolo puntuale dei fabbisogni di materie prime necessari per lo svolgimento della lezione;
    \item \textbf{Scelta delle Chiavi Primarie}: Esplicita le decisioni progettuali riguardanti l'adozione di identificatori naturali per garantire una maggiore aderenza semantica al dominio applicativo
\end{enumerate}
\subsubsection{Gestione Utenti e Gerarchie}
Il nucleo della gestione degli accessi e delle anagrafiche è rappresentato dall'entità padre \textbf{Utente}. Questa entità fattorizza le informazioni comuni a tutti gli attori del sistema, evitando ridondanze.
\begin{itemize}
    \item \textbf{Attributi Comuni}: Ogni utente è definito da \textit{Nome}, \textit{Cognome}, \textit{Email} e \textit{Password};
    \item \textbf{Gerarchia di Generalizzazione}: Il modello adotta una struttura gerarchica per gestire i ruoli. La relazione padre-figlio è definita dalle seguenti proprietà vincolanti:
    \begin{itemize}
        \item \textbf{Totale}: Indica che l'unione delle entità figlie copre l'intera popolazione dell'entità padre. In termini operativi, non può esistere un "Utente generico" nel sistema; ogni account registrato deve necessariamente appartenere ad almeno una delle categorie specifiche (\textbf{Studente} o \textbf{Chef});
        \item \textbf{Overlapping} (Cerchio con "O"): Questa configurazione permette la sovrapposizione dei ruoli. Un singolo utente può esistere simultaneamente sia come \textbf{Studente} che come \textbf{Chef} (configurando il ruolo ibrido di "\textbf{ChefStudente}"). Questa scelta garantisce flessibilità, permettendo ad esempio a uno chef di iscriversi a corsi tenuti da colleghi per aggiornamento professionale.
    \end{itemize}
    \item Le entità figlie estendono il padre con attributi specifici:
    \begin{itemize}
        \item \textbf{Studente}: Identificato univocamente nel contesto accademico dalla \textit{matricola};
        \item \textbf{Chef}: Qualificato professionalmente dall'attributo \textit{specializzazione}.
    \end{itemize}
\end{itemize}

\subsubsection{Corsi e Iscrizioni}
L'entità \textbf{Corso} rappresenta l'unità centrale dell'offerta didattica, fungendo da aggregatore per \textbf{Chef}e \textbf{Studente}. La sua esistenza è regolata da due relazioni fondamentali che definiscono, rispettivamente, la titolarità dell'insegnamento e la fruizione del servizio.
\begin{itemize}
    \item Relazione \textbf{Gestisce} (Titolarità del Corso): Questa associazione collega l'entità \textbf{Chef} a \textbf{Corso}:
    \begin{itemize}
        \item \textbf{Chef → Corso (0..N) }: Uno \textbf{Chef} può non avere corsi attivi in un determinato momento (0) oppure gestirne molteplici contemporaneamente (N);
        \item \textbf{Corso → Chef (1..N) - Vincolo di Esistenza)}: Un \textbf{Corso}, per essere creato e mantenuto nel sistema, necessita obbligatoriamente di un responsabile. Il vincolo minimo "1" impedisce la presenza di "corsi orfani" senza docente, garantendo la qualità dell'offerta.
    \end{itemize}
    \item Relazione \textbf{Iscrizione} (Fruizione Didattica): Collega l'entità \textbf{Studente} a \textbf{Corso}. Trattandosi di una relazione \textbf{Molti-a-Molti}, essa evolve in una tabella associativa dotata di attributi propri che storicizzano il legame:
    \begin{itemize}
        \item Attributi della relazione:
        \begin{itemize}
            \item \textit{dataIscrizione}: Traccia temporalmente il momento dell'ingresso dello studente nel corso;
            \item \textit{stato}: Definisce la fase corrente del percorso (es. Attivo, Completato, Ritirato), permettendo di monitorare il progresso accademico.
        \end{itemize}
        \item Cardinalità della relazione:
        \begin{itemize}
            \item \textbf{Studente → Iscrizione (0..N)}: Uno \textbf{Studente} registrato può non essere iscritto a nessun corso (0) o seguirne diversi in parallelo (N).
             \item \textbf{Corso → Iscrizione (1..N) - Vincolo di Attivazione)}: Similmente alla gestione degli \textbf{Chef}, il modello impone che un corso esista solo se vi è una platea di destinatari. Un corso con 0 iscritti non è contemplato operativamente come istanza attiva nel sistema.
        \end{itemize}
    \end{itemize}
\end{itemize}
\subsubsection{Struttura delle Sessioni}
Il \textbf{Corso} non è un'entità monolitica, ma viene declinato in unità didattiche temporali attraverso la relazione \textbf{Compone}. Questa associazione definisce la scomposizione modulare del percorso formativo in singole lezioni, permettendo una pianificazione granulare del calendario accademico.
\begin{itemize}
    \item \textbf{Relazione Compone} (Vincoli di Struttura): La relazione lega l'offerta formativa alla sua erogazione pratica. Le cardinalità imposte garantiscono la consistenza strutturale:
    \begin{itemize}
        \item \textbf{Corso → Sessioni (1..N)}: Un corso non può esistere come "scatola vuota"; affinché sia valido, deve essere composto da almeno una sessione pianificata;
        \item \textbf{Sessione → Corso (1..1}): Ogni sessione è strettamente dipendente dal corso di appartenenza. Non esistono sessioni "libere" o condivise tra più corsi; ogni lezione è un'istanza esclusiva del percorso formativo di riferimento.
    \end{itemize}
    \item \textbf{Entità Sessione} (Unità Atomica): La Sessione rappresenta l'unità base di erogazione del servizio didattico.
    \begin{itemize}
        \item Attributo \textit{oraInizio}: Fondamentale per la calendarizzazione, questo attributo temporale definisce il collocamento cronologico della lezione, permettendo la gestione degli orari.
    \end{itemize}
    \item \textbf{Gerarchia delle Sessioni} (Generalizzazione): Per gestire la natura ibrida della didattica, è stata modellata una gerarchia. Si tratta di una generalizzazione \textbf{Totale} e \textbf{Disgiunta} (cerchio con "D"):
    \begin{itemize}
        \item \textbf{Disgiunta}: Impone una mutua esclusività rigida. Un'istanza di \textbf{sessione} non può possedere simultaneamente le caratteristiche di una lezione online e di una in presenza. Lo stato è binario: o la lezione avviene da remoto, o avviene in laboratorio.
    \end{itemize}
    La gerarchia si dirama nelle seguenti entità figlie:
    \begin{itemize}
        \item \textbf{SessioneOnline}: Rappresenta le lezioni erogate tramite piattaforme digitali.
        \begin{itemize}
            \item Attributo \textit{teoria}: Questo attributo specifico qualifica il contenuto della lezione (es. link alla piattaforma, argomento teorico trattato), focalizzandosi esclusivamente sulla componente concettuale e nozionistica, svincolata dalle necessità logistiche fisiche.
        \end{itemize}
        \item \textbf{SessionePresenza} (Laboratorio Fisico): Rappresenta le lezioni pratiche svolte all'interno delle strutture dell'istituto.
        \begin{itemize}
            \item Questa entità funge da snodo cruciale per l'intera gestione logistica (descritta nella sezione successiva), poiché è l'unico punto del sistema che abilita le associazioni con gli ingredienti, le ricette e la presenza fisica degli studenti.
            \item Attributo Derivato Multivalore (\textit{quantitàTotale}): A livello concettuale, è stato definito l’attributo \textit{quantitàTotale} per rappresentare il fabbisogno complessivo di materie prime. Esso è classificato come multivalore (in quanto riferito a una lista eterogenea di ingredienti) e derivabile (rappresentato graficamente con linea tratteggiata).
        \end{itemize}
    \end{itemize}
\end{itemize}

\subsubsection{Logistica in Presenza (Adesioni e Fabbisogni)}
Questa porzione del diagramma modella le dinamiche operative che avvengono fisicamente all’interno del laboratorio, distinguendo le attività pratiche dal semplice apprendimento teorico. Qui vengono gestiti i flussi di prenotazione degli studenti e la definizione puntuale delle materie prime necessarie.
\begin{itemize}
    \item \textbf{Adesione degli Studenti}: A differenza della generica iscrizione al \textbf{Corso}, la relazione \textbf{Adesione} rappresenta la prenotazione puntuale di uno studente a una specifica lezione pratica (\textbf{SessionePresenza}).
    \begin{itemize}
        \item Attributo \textit{dataAdesione}: Registra il momento esatto della prenotazione, fondamentale per gestire priorità o scadenze temporali per la partecipazione;
        \item Attributo Derivato \textit{numAderenti}: La SessionePresenza include questo attributo calcolato, che conta dinamicamente le istanze nella relazione \textbf{Adesione} associate a quella \textbf{sessione}.
    \end{itemize}
    \item \textbf{Definizione dei Fabbisogni (Relazione Richiede)}: La relazione \textbf{Richiede} costituisce la distinta base tecnica del laboratorio, collegando ogni \textbf{Ricetta} ai singoli \textbf{Ingredienti} necessari per la sua realizzazione. Questa associazione è fondamentale per tradurre un piatto in una lista di spesa.
    \begin{itemize}
        \item Attributo \textit{quantitàNecessaria}: Rappresenta la dose unitaria effettiva dell'ingrediente per la specifica ricetta . Questo attributo è il moltiplicatore base che, incrociato successivamente con il numero di aderenti alla sessione, permetterà al sistema di calcolare la \textit{quantitàTotale} (attributo derivato multivalore).
        \item \textbf{Analisi delle Cardinalità (Vincoli di Struttura)}: Le cardinalità imposte sulla relazione \textbf{Richiede }garantiscono la consistenza delle schede tecniche:
        \begin{itemize}
            \item \textbf{Ricetta → Ingredienti (1..N)}: Una \textbf{ricetta }non può esistere come entità astratta priva di componenti; affinché sia valida e "cucinabile", deve essere composta da almeno un \textbf{ingrediente};
            \item \textbf{Ingrediente → Ricette (0..N)}: Un \textbf{ingrediente }catalogato in magazzino (es. una spezia rara) potrebbe momentaneamente non essere utilizzato in nessuna delle ricette attive nel menu corrente (cardinalità minima 0), ma può potenzialmente comparire in infinite preparazioni diverse (cardinalità massima N).
        \end{itemize}
    \end{itemize}
\end{itemize}
\subsubsection{Scelta delle Chiavi Primarie (Identificatori Naturali)}
Nel progettare lo schema, si è scelto di utilizzare, ove possibile, \textbf{chiavi naturali} (proprietà inerenti all'entità) anziché chiavi surrogate, al fine di mantenere una forte coerenza concettuale con il dominio applicativo.
\begin{itemize}
    \item \textbf{Email (Entità Utente)}: È stata selezionata l'attributo \textit{email} come chiave primaria per l'entità \textbf{Utente}. Nel contesto di una piattaforma web, l'indirizzo email costituisce un identificatore univoco globale che garantisce l'assenza di duplicati per la stessa persona fisica. Questa scelta ottimizza anche il processo di autenticazione, poiché l'email funge sia da identificativo di accesso che da chiave di vincolo nel database;
    \item \textbf{Nome (Entità Ingrediente)}: Per l'entità \textbf{Ingrediente}, si è scelto l'attributo \textit{nome} come chiave primaria. Si assume che nel dominio di riferimento (la dispensa del laboratorio) ogni ingrediente sia catalogato con una nomenclatura univoca (es. "Farina", "Zucchero a velo"). L'uso del nome come chiave evita ridondanze e semplifica le interrogazioni, rendendo le associazioni immediatamente leggibili senza necessità di operazioni di join per recuperare la descrizione dell'ingrediente.
\end{itemize}
\clearpage
\subsection{Diagramma UML}
\begin{figure}[!h] % [h] = here, posizione suggerita
    \centering
   \includegraphics[width=1.0\textwidth]{CartellaDocumentazione/diagrammi/DiagrammaUML1.pdf}
\end{figure}
\subsubsection{Descrizione Diagramma UML}
\subsubsection{Tipi di Dato e Precisione Numerica}
Il diagramma UML definisce esattamente come i dati devono essere memorizzati, sciogliendo le ambiguità generiche dell'ER:
\begin{itemize}
\item \textbf{Gestione delle Quantità}: Si nota una distinzione importante tra Integer e Decimal;
\item Contatori come \textit{numPartecipanti} e \textit{numAderenti} sono Int, poiché contano unità discrete (persone, lezioni);
\item Le misure fisiche come \textit{quantitàTotale} e \textit{quantitàNecessaria}  sono definite come Decimal. Questo indica che il sistema gestirà valori decimali precisi (es. 1.5 kg, 0.25 litri), dettaglio fondamentale per le ricette che l'ER lasciava implicito.
\item \textbf{Date e Orari}: i riferimenti temporali sono distinti in base al significato: \textit{dataAdesione} e \textit{dataIscrizione} sono Date, \textit{oraInizio} è un DateTime (data+ora), mentre \textit{tempo} è un Time (ora, minuti, secondi).
\end{itemize}
\subsubsection{Rappresentazione degli Array (Attributi Multivalore)}
L’UML traduce graficamente gli attributi multivalore dell’ER con la notazione delle parentesi quadre: 
\begin{itemize}
    \item +\textit{specializzazione}[1...*]: String nella classe \textbf{Chef};
    \item +\textit{categoria}[1...*]: String nella classe \textbf{Corso}.
\end{itemize}
Questa notazione dice esplicitamente che posso avere più valori.\\
Un caso particolare è l'attributo della logistica: \textit{«derivate» quantitàTotale}[1...*]: nella classe \textbf{SessionePresenza}. In questo caso, la notazione evidenzia due aspetti:
\begin{itemize}
    \item Lo stereotipo «derivate»: indica che il dato non viene memorizzato, ma è calcolato dinamicamente dal sistema (aggregando le ricette).
    \item Le parentesi [1...*]: specificano che il risultato del calcolo non è un valore unico, ma una lista di ingredienti e quantità.
\end{itemize}
\clearpage

\section{Progettazione Logica}
\subsection{Ristrutturazione Diagramma UML}
\begin{figure}[!h] % [h] = here, posizione suggerita
    \centering
    \includegraphics[width=1.0\textwidth]{CartellaDocumentazione/diagrammi/DiagrammaUMLRistrutt1.pdf}
\end{figure}

\subsubsection{Descrizione Delle Scelte}
Nel passaggio dal modello concettuale a quello logico, si è scelto di semplificare la struttura delle gerarchie ereditarie adottando la strategia dell'accorpare nella classe genitore. Questa decisione mira a ridurre la complessità dello schema e a ottimizzare le prestazioni, evitando join eccessivi per recuperare informazioni su entità strettamente correlate.
\subsubsection{Accorpamento della gerarchia Utente}
Nel modello iniziale, l'entità \textbf{Utente} si specializzava nelle sottoclassi \textbf{Studente} e \textbf{Chef}. L'analisi ha evidenziato che la sottoclasse \textbf{Studente} possedeva un set di attributi propri molto ridotto, rendendo una separazione strutturale su tabelle distinte poco efficiente. Si è proceduto quindi alla fusione in un'unica entità \textbf{Utente}, gestendo le specificità tramite le seguenti modifiche:
\begin{itemize}
    \item \textbf{Attributo Discriminante (Enum)}: È stato introdotto il campo \textit{TipoUtente} basato su un'enumerazione. I valori previsti sono \textit{chef}, \textit{studente} e \textit{chefStudente}. Quest'ultimo valore è cruciale per implementare logicamente la generalizzazione overlapping, permettendo al sistema di identificare gli utenti che ricoprono entrambi i ruoli senza duplicare i record.
    \item Gestione degli Attributi Specifici:
    \begin{itemize}
        \item \textit{matricola} [0..1]: L'attributo scalare è stato integrato nella tabella \textbf{Utente} con vincolo di nullabilità (nullable). Sarà valorizzato esclusivamente se il \textit{TipoUtente} include il ruolo di \textit{studente} o \textit{chefStudente};
        \item 
        \textit{specializzazione} (Esternalizzazione): A differenza dell'attributo \textit{matricola}, l'attributo \textit{specializzazione} presentava una cardinalità multivalore (uno \textbf{Chef} può avere più competenze). Per rispettare la giusta gestione degli attributi multivalori, non è stato accorpato nell'entità \textbf{Utente}, ma è stato promosso a relazione indipendente (tabella \textbf{Specializzazione\_Chef}), collegata all'utente tramite chiave esterna.
    \end{itemize}
\end{itemize}

\subsubsection{Accorpamento della gerarchia Sessione }
Analogamente, la generalizzazione tra \textbf{SessioneOnline} e \textbf{SessionePresenza} è stata risolta accorpando le sottoclassi nell'entità padre \textbf{Sessione}. La scarsa complessità informativa della classe \textbf{SessioneOnline} non giustificava il mantenimento di una tabella dedicata.
\begin{itemize}
    \item \textbf{Attributo Discriminante (Enum)}: È stato introdotto il campo TipoSessione con i valori \textit{online }e \textit{presenza}. Poiché la generalizzazione originale era disgiunta, questo attributo assume un valore mutuamente esclusivo.
    \item \textbf{Integrazione degli Attributi Condizionali}: La tabella unica \textbf{Sessione} accoglie ora l'unione degli attributi delle sottoclassi originali, resi opzionali per garantirne la coerenza semantica:
    \begin{itemize}
        \item \textit{teoria }[0..1]: Attributo testuale valorizzato solo se l'attributo discriminante \textit{TipoSessione} è \textit{online};
        \item \textbf{Attributi Logistici (\textit{numAderenti}, \textit{quantitàTotale})}: Questi attributi sono pertinenti e validi esclusivamente se l'attributo discriminante \textit{TipoSessione} è \textit{presenza}. In caso di sessione online, tali campi restano nulli o non calcolabili, riflettendo l'assenza di logistica fisica.
    \end{itemize}
\end{itemize}
\subsubsection{Gestione degli Attributi Multivalore (Normalizzazione)}
Durante la fase di traduzione dal modello concettuale al modello logico relazionale, è stata posta particolare attenzione alla gestione degli attributi definiti come \textbf{multivalore}. Il modello relazionale impone il rispetto della \textbf{Prima Forma Normale (1FN)}, la quale prescrive l'atomicità dei valori: ogni cella di una tabella deve contenere un singolo valore e non sono ammessi gruppi ripetitivi o array.\\
Per risolvere questa discrepanza strutturale, si è proceduto alla promozione degli attributi multivalore, trasformandoli in relazioni (tabelle) distinte collegate all'entità principale tramite vincoli di chiave esterna.
\begin{enumerate}
    \item \textbf{Esternalizzazione delle Specializzazioni Chef}\\
Nel modello concettuale, l'entità \textbf{Chef} possedeva l'attributo \textit{specializzazione} con cardinalità [1..N], indicando che un docente può possedere competenze multiple. L'integrazione di questo dato direttamente nella tabella \textbf{Utente} avrebbe generato una ridondanza dei dati anagrafici (ripetendo le informazioni dell'utente per ogni specializzazione) o la creazione di campi non atomici, compromettendo l'integrità del database e violando la Prima Forma Normale. Quindi si è optato per la creazione della tabella dedicata \textbf{Specializzazione\_Chef}, composta dalla coppia di attributi  (\textit{email\_chef, specializzazione}) e avente una chiave primaria composta da entrambi gli attributi, garantendo che non ci siano duplicati della stessa competenza per lo stesso chef.
    \item \textbf{Esternalizzazione delle Categorie Corso}\\
Analogamente, l'attributo \textit{categoria} associato all'entità \textbf{Corso} è stato identificato come multivalore (un corso può appartenere a più aree tematiche simultaneamente, es. "Primi Piatti" e "Tradizione Romana"), perciò è stata introdotta la tabella di collegamento \textbf{Categoria\_Corso}, composta dalla coppia di attributi (\textit{id\_corso}, \textit{categoria}) e avente una chiave primaria composta da entrambi gli attributi.\\
Questa struttura permette una classificazione flessibile dell'offerta formativa, consentendo ricerche filtrate per categoria senza appesantire la tabella principale Corso con campi testuali ridondanti.
\end{enumerate}
\subsubsection{Gestione dell'attributo derivato multivalore quantitàTotale}
Nella fase di ristrutturazione del diagramma per il passaggio al modello logico-relazionale, è stata posta particolare attenzione all'attributo \textit{quantitàTotale}, definito nel modello concettuale come derivato e multivalore. La traduzione diretta di tale attributo all'interno della tabella \textbf{Sessione} avrebbe comportato due gravi violazioni delle forme normali:
\begin{itemize}
    \item \textbf{Violazione della 1FN (Atomicità)}: Trattandosi di un attributo multivalore (una lista di ingredienti diversi per ogni sessione), non può essere memorizzato in un'unica cella atomica.
    \item \textbf{Ridondanza e Rischio di Inconsistenza}: Memorizzare un valore calcolato (il totale) che dipende da altri dati già presenti nel database (\textit{quantitàNecessaria} in \textbf{Richiede} e \textit{numAderenti} in \textbf{Adesione}) introdurrebbe una ridondanza. Se una dose venisse modificata, il totale memorizzato diventerebbe obsoleto (anomalia di aggiornamento).
\end{itemize}
Quindi si è optato all'utilizzo di una \textbf{Vista Logica (VIEW}).\\
Per risolvere queste criticità mantenendo l'accesso agevole al dato aggregato, l'attributo è stato rimosso dallo schema fisico delle tabelle ed è stato implementato tramite una Vista SQL (Virtual Relation).La vista, denominata \textbf{Vista\_Fabbisogni\_Sessione}, non memorizza fisicamente i dati, ma esegue una query di aggregazione a runtime che incrocia le seguenti entità:
\begin{itemize}
    \item \textbf{Sessione}: Fornisce il moltiplicatore (\textit{NumAderenti}).
    \item \textbf{Prepara (Tabella di collegamento Sessione-Ricetta)}: Funge da filtro, identificando quali ricette sono effettivamente previste per quella specifica sessione. Senza questa tabella ponte, sarebbe impossibile sapere quale "lista ingredienti" attivare per la lezione.
    \item \textbf{Richiede}: Fornisce la dose unitaria (\textit{quantitàNecessaria}) per ogni ingrediente della ricetta selezionata.
\end{itemize}





\subsubsection{Adozione delle Chiavi Surrogate (Identificativi Artificiali)}
Mentre per le entità anagrafiche e stabili (come \textbf{Utente} e \textbf{Ingrediente}) si è optato per chiavi naturali, per le entità associative e di gestione eventi si è ritenuto necessario introdurre \textbf{chiavi surrogate} (o artificiali). Tali chiavi sono costituite da codici numerici autoincrementali privi di significato semantico intrinseco.\\
La scelta di utilizzare identificativi artificiali per entità come \textbf{Corso}, \textbf{Sessione} e \textbf{Ricetta} è motivata da tre criteri progettuali fondamentali:
\begin{enumerate}
    \item \textbf{Stabilità e Indipendenza Semantica}\\
    Una Chiave Primaria deve soddisfare il requisito di immutabilità. Utilizzare attributi descrittivi (come il "Titolo del Corso" o il "Nome della Ricetta") come chiave primaria esporrebbe il database a gravi problemi di integrità referenziale nel caso in cui tali nomi dovessero subire modifiche o correzioni.
    \item \textbf{Efficienza nelle Operazioni di Join}\\
    Dal punto di vista delle prestazioni, i database relazionali gestiscono i confronti tra numeri interi (INTEGER) in modo significativamente più rapido rispetto al confronto tra stringhe di testo (VARCHAR). Poiché le tabelle operative (come \textbf{Iscrizione} o \textbf{Richiede}) contengono migliaia di record che fanno riferimento ai corsi e agli ingredienti, l'uso di chiavi intere riduce la dimensione degli indici e velocizza le operazioni di JOIN.
    \item \textbf{Semplificazione delle Chiavi Esterne}\\
    Per alcune entità deboli o dipendenti, l'uso di chiavi naturali avrebbe richiesto la creazione di chiavi composte complesse. Esempio:
    \begin{itemize}
        \item  \textbf{Sessione}: Senza una chiave surrogata, per identificare univocamente una sessione sarebbe stato necessario combinare \textit{id\_corso} +  \textit{OraInizio}. Utilizzando un semplice \textit{id\_sessione} univoco, si semplifica drasticamente la struttura delle tabelle associative (come \textbf{Adesione} e \textbf{Prepara}), che necessitano di importare una sola colonna numerica come chiave esterna anziché due.
    \end{itemize}
\end{enumerate}


\subsection{Dizionari}
\subsubsection{Dizionario delle Classi}

\begin{table}[H] % oppure [htbp]
\label{tab:classi1}

\centering
\resizebox{0.9\textwidth}{!}{
\begin{tabular}{|>{\centering\arraybackslash}m{0.22\textwidth}|>
                {\centering\arraybackslash}m{0.28\textwidth}|>{\raggedright\arraybackslash}m{0.45\textwidth}|}

                \hline
    \textbf{Classe} & \textbf{Descrizione} & \textbf{Attributi} \\
    \hline
    \multirow{12}{=}{\centering Utente} & \multirow{12}{=}{\centering Descrittore di ogni Studente o Chef } &
\texttt{email} [PK](String): è un indirizzo email unico e valido che permette di contattare l’utente.\\

& & \texttt{nome} (String): è il nome dell'utente.\\

& & \texttt{cognome} (String): è cognome dell'utente. \\

& & \texttt{password} (String): una password permette all'utente di proteggere il suo account e le sue informazioni.\\

& & \texttt{matricola} (String): string alfanumerica che identifica lo studente. \\ 

& & \texttt{tipoUtente} (TipoUtente): indica il tipo di utente: (chef), (studente) oppure (chefStudente). \\


\hline
\multirow{8}{=}{\centering Corso} & \multirow{8}{=}{\centering Descrittore di ciascun corso creato} &

\texttt{id\_corso} [PK] (Int): identificativo univoco di Corso.\\

&& \texttt{dataInizio} (Data): indica l'inizio del corso.\\

& & \texttt{nome} (String): è il nome assegnato al corso. \\

& & \texttt{frequenza} (String): indica ogni quanti giorni il corso si svolge.\\

& & \texttt{numPartecipanti} (Int): Numero dei partecipanti al corso. \\
& & \texttt{numSessioni} (Int): specifica quante sessioni ha un corso. \\
\hline

\multirow{3}{=}{\centering Iscrizione} & \multirow{3}{=}{\centering Descrittore dell'iscrizione degli utenti} &
\texttt{dataIscrizione} (Data): indica l'inizio dell'iscrizione per un corso.\\

& & \texttt{Stato} (String): indica lo stato dell'iscrizione di uno studente. \\
\hline

\end{tabular}
}

\end{table}




\begin{table}[H] % oppure [htbp]
\label{tab:classi2}
\centering
\resizebox{0.9\textwidth}{!}{
\begin{tabular}{|>{\centering\arraybackslash}m{0.22\textwidth}|>
                {\centering\arraybackslash}m{0.28\textwidth}|>{\raggedright\arraybackslash}m{0.45\textwidth}|}

                \hline
    \textbf{Classe} & \textbf{Descrizione} & \textbf{Attributi} \\
    \hline
    \multirow{12}{=}{\centering Sessione} & \multirow{12}{=}{\centering Descrittore di ciascuna sessione presente in un corso} & 
    \texttt{id\_sessione} [PK] (Int): identificativo univoco di Sessione\\
    &&\texttt{oraInizio} (Data): orario di inizio della sessione.\\

& & \texttt{numAderenti} (Int): numero di partecipanti alla sessione. \\

& & \texttt{teoria} (String): argomento teorico spiegato in sessione.\\

& & \texttt{tipoSessione} (TipoSessione): indica il tipo di sessione: (online) oppure (presenza). \\ 



\hline
\multirow{6}{=}{\centering Ricetta} & \multirow{6}{=}{\centering Descrittore di ciascuna ricetta usata in una sessione} &
\texttt{id\_ricetta} [PK] (Int): identificativo univoco di Ricetta.\\

&& \texttt{nome} (String): nome della ricetta.\\

& & \texttt{descrizione} (String): la preaparazione della ricetta. \\

& & \texttt{tempo} (Time): tempo per la preparazione della ricetta. \\


\hline

\multirow{4}{=}{\centering Ingrediente} & \multirow{4}{=}{\centering Descrittore di ciascun ingrediente usato per una ricetta} &
\texttt{nome} [PK] (String): nome dell'ingrediente.\\

& & \texttt{unitàDiMisura} (String): specifica dell'unità di misura di un ingrediente ES: Kg, litri ecc.. .\\

\hline


\centering Adesione & \centering Classe associativa per gestire l'associazione tra Utente (studente/chefStudente) e Sessione (presenza) &
\texttt{dataAdesione} (Data): data relativa all'adesione di uno studente ad una sessione.\\


\hline

\centering Richiede & \centering Classe associativa per gestire l'associazione tra Ingrediente e Ricetta &
\texttt{quantitàNecessaria} (Decimal): quantità utilizzata di ogni singolo ingrediente per una ricetta.\\

\hline

\centering Categoria\_Corso & \centering Classe per gestire l'attributo multivalore categoria del corso &
{\ttfamily categoria} (String): una categoria associata a un corso.\\

\hline

\centering Specializzazione\_ \space Chef & \centering Classe per gestire l'attributo multivalore specializzazione dello chef. &
{\ttfamily specializzazione} (String): una specializzazione associata a uno chef.\\

\hline

\end{tabular}
}

\end{table}

\subsubsection{Dizionario Associazioni}


\begin{table}[H] % oppure [htbp]
\label{tab:classi} 
\centering
%\resizebox{1.\textwidth}{!}{
\begin{tabular}{|>{\centering\arraybackslash}m{0.22\textwidth}|>
                {\centering\arraybackslash}m{0.28\textwidth}|>{\raggedright\arraybackslash}m{0.45\textwidth}|}
                \hline
    \textbf{Associazione} & \textbf{Descrizione} & \textbf{Molteplicità delle Classi coinvolte} \\
    \hline
  \centering richiede (adesione) & \centering Richiesta di adesione da parte di uno studente & Utente [0..*] - Adesione [1] \\



\hline

  \centering richiede (iscrizione) & \centering Richiesta di iscrizione da parte di uno studente & Utente [0..*] - Iscrizione [1] \\

\hline

  \centering gestisce & \centering Gestione di un corso da parte di uno chef & Utente [0..*] - Corso [1..*] \\
\hline

  \centering compone & \centering Un corso è composto da più sessioni  & Corso [1] - Sessione [1..*] \\
\hline  



  \centering prepara & \centering Ad ogni sessione in presenza si prepara una ricetta & Sessione [1..*] - Ricetta [1..*] \\

\hline

  \centering haBisogno & \centering ogni ricetta ha bisogno di sapere la quantità degli ingredienti  & Ricetta [1] - Richiede [1..*] \\

\hline


  \centering necessita & \centering Un ingrediente necessita di sapere la quantità per la ricetta & Richiede [1..*] - Ingrediente [1] \\

\hline

  \centering per & \centering Richiesta di adesione per una sessione in presenza & Adesione [1] - Sessione [0..*] \\

\hline

  \centering relativa & \centering Iscrizione di uno studente per un corso & Iscrizione [1] - Corso [1..*] \\

\hline 
   \centering specializzatoIn & \centering Specializzazione di uno chef & Utente [1] - Specializzazione\_Chef [0..*] \\

\hline
   \centering categorizzatoIn & \centering Categoria a cui appartiene un corso & Corso [1] - Categoria\_Corso [1..*] \\

\hline

\end{tabular}
%}

\end{table}
\subsubsection{Dizionario delle Viste}

\begin{table}[H] % oppure [htbp]
\label{tab:classi} 
\centering
%\resizebox{1.\textwidth}{!}{
\begin{tabular}{|>{\centering\arraybackslash}m{0.22\textwidth}|>
                {\centering\arraybackslash}m{0.28\textwidth}|>{\raggedright\arraybackslash}m{0.45\textwidth}|}
                \hline
    \textbf{Viste} & \textbf{Descrizione} & \textbf{Attributi} \\
    \hline
  \centering Vista\_Fabbisogni\_\\Sessione & \centering Struttura virtuale che aggrega i dati logistici. Incrocia le adesioni e le ricette per calcolare il fabbisogno puntuale di materie prime & quantità\_totale: (Derivato) Valore numerico che esprime il fabbisogno complessivo dello specifico ingrediente per la specifica sessione (calcolato come: Dose Unitaria × Numero Partecipanti). \\
  \hline

\end{tabular}
%}
\end{table}
\subsubsection{Dizionario Vincoli}


\renewcommand{\arraystretch}{1.3}

\begin{center}
\begin{longtable}{|>{\centering\arraybackslash}m{0.30\textwidth}|
                  >{\centering\arraybackslash}m{0.22\textwidth}|
                  >{\raggedright\arraybackslash}m{0.45\textwidth}|}
\caption{Dizionario dei vincoli} \label{tab:vincoli} \\

\hline
\textbf{Vincolo} & \textbf{Tipo} & \textbf{Descrizione} \\
\hline
\endfirsthead

\hline
\textbf{Vincolo} & \textbf{Tipo} & \textbf{Descrizione} \\
\hline
\endhead

Email & Dominio & L’attributo “email” della classe Utente deve contenere una combinazione non nulla di lettere, numeri e simboli, seguiti da una chiocciola (“@”), altre lettere/simboli, un punto (“.”), e finire con almeno due lettere. \\
\hline

Password & Dominio & L’attributo “password” della classe Utente deve essere lungo almeno 8 caratteri, e avere almeno una lettera maiuscola, una minuscola, un numero e un carattere speciale. \\
\hline

Nome & Dominio & L’attributo “Nome” della classe Utente deve contenere solo stringhe di caratteri alfabetici, non vuote e con lunghezza <= 50 caratteri. \\
\hline

Cognome & Dominio & L’attributo “Cognome” della classe Utente deve contenere solo stringhe di caratteri alfabetici, non vuote e con lunghezza <= 50 caratteri. \\
\hline

TipoUtente & Dominio & L’attributo "tipoUtente" della classe Utente è un’enumerazione che può assumere solo i valori: chef, studente, chefStudente, per identificare il ruolo dell’utente nel sistema. \\
\hline

Specializzazione & N‑tupla & Se l’attributo “TipoUtente” è chef o chefStudente, allora l’attributo “Specializzazione” assume un valore alfabetico con lunghezza <= 50. \\
\hline

Matricola & N‑tupla & Se l’attributo “TipoUtente” è studente o chefStudente, allora l’attributo “Matricola” assume un valore alfanumerico con lunghezza <= 50. \\
\hline

DataInizio & Dominio & L'attributo "DataInizio" della classe Corso deve essere una data valida nel formato previsto, non può essere nulla. \\
\hline

Nome & Dominio & L'attributo "Nome" della classe Corso deve contenere una stringa non vuota con solo caratteri alfabetici e lunghezza massima 50 caratteri. \\
\hline

Categoria & N‑tupla & L'attributo "Categoria" della classe Corso deve contenere almeno una categoria. Ogni elemento deve essere una stringa non vuota. \\
\hline

Frequenza & Dominio & L'attributo "Frequenza" della classe Corso deve contenere una stringa che rappresenta un intervallo temporale coerente. \\
\hline

NumPartecipanti & Intrarelazionale & L'attributo "NumPartecipanti" della classe Corso è calcolato automaticamente in base al numero di iscrizioni associate. \\
\hline

NumSessioni & Dominio & L'attributo "NumSessioni" della classe Sessione deve essere un intero positivo >0. \\
\hline

OraInizio & Dominio & L'attributo "OraInizio" della classe Sessione deve essere una data valida, non nulla. \\
\hline

TipoSessione & Dominio & L'attributo "TipoSessione" della classe Sessione è un’enumerazione che può assumere solo i valori: presenza, online. \\
\hline

Quantità & Intrarelazionale & Se l'attributo "TipoSessione" è "presenza", allora "Quantità" assume un valore numerico >0. \\
\hline

NumAderenti & Intrarelazionale & Se l'attributo "TipoSessione" è "presenza", allora "NumAderenti" assume un valore numerico valido >0. \\
\hline

Teoria & Intrarelazionale & Se l'attributo "TipoSessione" è "online", allora "Teoria" deve essere una stringa alfanumerica con lunghezza <=255. \\
\hline

Nome & Dominio & L'attributo "Nome" della classe Ricetta deve contenere una stringa alfabetica non vuota con lunghezza <=50. \\
\hline

Descrizione & Dominio & L'attributo "Descrizione" della classe Ricetta deve contenere una stringa alfabetica non vuota con lunghezza <=50. \\
\hline

Tempo & Dominio & L'attributo "Tempo" della classe Ricetta deve contenere una data (ora, minuti, secondi) valida e non vuota. \\
\hline

Nome & Dominio & L'attributo "Nome" della classe Ingrediente deve contenere una stringa alfabetica non vuota con lunghezza <=50. \\
\hline

UnitàDiMisura & Dominio & L'attributo "UnitàDiMisura" della classe Ingrediente deve contenere una stringa alfabetica non vuota con lunghezza <=50. \\
\hline

\end{longtable}
\end{center}


\section{Schema Logico}
\subsection{Tabelle}
Leggenda:
\begin{itemize}
    \item Le parole con il \textbf{grassetto} sono indicate tutti i nomi delle tabelle
    \item Le parole \underline {sottolinete} una volta indicano le chiavi primarie
    \item Le parole \underline{\underline{sottolinate}} due volte indicano le chiavi esterne
    \item Le parole in \textit{italico} indicano che una chiava primaria è composta da piu attributi 
\end{itemize}

\vspace{2em}

\renewcommand{\arraystretch}{1.80}
\begin{tabular}{lp{0.7\textwidth}} \hline
    \textbf{Utente} & email, nome, cognome, password, specializzazione, matricola, tipoUtente, id\_corso\\ 
    \hline
    \textbf{Corso} & id\_corso, dataInizio, nome, categoria, frequenza, numPartecipanti, id\_sessione \\
    \hline
    \textbf{Sessione} & id\_sessione, numSessioni, oraInizio, numAderenti, quantità, teoria, TipoSessione, id\_ricetta, id\_corso\\
    \hline
    \textbf{Ricetta} & id\_ricetta, nome, descrizione, tempo, id\_sessione \\
    \hline
    \textbf{Ingrediente} & nome, unitàDiMisura\\
    \hline
    \textbf{Adesione} & dataAdesione, email,id\_sessione\\
    \hline
    \textbf{Iscrizione} & dataIscrizione, stato, email,id\_corso\\
    \hline
    \textbf{Richiede} & quantitàNecessaria, id\_ricetta, nomeIngrediente \\
    \hline
\end{tabular}

\section{Implementazione della Base di Dati}
In questo capitolo viene illustrata la fase di implementazione fisica del database, derivata direttamente dallo schema logico precedentemente definito. La codifica è stata realizzata utilizzando il linguaggio SQL (dialetto PostgreSQL), ponendo particolare attenzione alla robustezza e alla consistenza dei dati.
\\
Oltre alla definizione delle strutture tabellari, una parte centrale dello sviluppo ha riguardato la traduzione delle regole di business in vincoli di integrità attivi. Sono stati implementati domini personalizzati, vincoli CHECK per la validazione dei formati e, soprattutto, TRIGGER  per garantire il rispetto delle cardinalità minime e delle logiche di aggiornamento automatico.

\subsection{Creazione e pulizia schema}
\begin{figure}[h!]
    \centering 
    \includegraphics[width=0.8\textwidth]{CartellaDocumentazione/immagini/creazione schema.png}
\end{figure}
 \subsection{Creazione delle enumerazioni}
\begin{figure}[h!]
    \centering 
    \includegraphics[width=0.8\textwidth]{CartellaDocumentazione/immagini/creazione enum.png}
\end{figure}
\clearpage
\subsection{Creazione delle tabelle}
\subsubsection{Tabella Utente}
\begin{figure}[h!]
    \centering 
    \includegraphics[width=0.8\textwidth]{CartellaDocumentazione/tabelle/tabella Utente.png}
\end{figure}
\subsubsection{Tabella Corso}
\begin{figure}[h!]
    \centering 
    \includegraphics[width=0.8\textwidth]{CartellaDocumentazione/tabelle/tabella Corso.png}
\end{figure}
\subsubsection{Tabella Sessione}
\begin{figure}[h!]
    \centering 
    \includegraphics[width=0.8\textwidth]{CartellaDocumentazione/tabelle/tabella Sessione.png}
\end{figure}
\clearpage
\subsubsection{Tabella Ricetta}
\begin{figure}[h!]
    \centering 
    \includegraphics[width=0.8\textwidth]{CartellaDocumentazione/tabelle/tabella Ricetta.png}
\end{figure}
\subsubsection{Tabella Ingrediente}
\begin{figure}[h!]
    \centering 
    \includegraphics[width=0.8\textwidth]{CartellaDocumentazione/tabelle/tabella Ingrediente.png}
\end{figure}
\subsubsection{Tabella Iscrizione}
\begin{figure}[h!]
    \centering 
    \includegraphics[width=0.8\textwidth]{CartellaDocumentazione/tabelle/tabella Iscrizione.png}
\end{figure}
\clearpage
\subsubsection{Tabella Adesione}
\begin{figure}[h!]
    \centering 
    \includegraphics[width=0.8\textwidth]{CartellaDocumentazione/tabelle/tabella Adesione.png}
\end{figure}
\subsubsection{Tabella Gestisce}
\begin{figure}[h!]
    \centering 
    \includegraphics[width=0.8\textwidth]{CartellaDocumentazione/tabelle/tabella Gestisce.png}
\end{figure}
\subsubsection{Tabella Prepara}
\begin{figure}[h!]
    \centering 
    \includegraphics[width=0.8\textwidth]{CartellaDocumentazione/tabelle/tabella Prepara.png}
\end{figure}
\subsubsection{Tabella Richiede}
\begin{figure}[h!]
    \centering 
    \includegraphics[width=0.8\textwidth]{CartellaDocumentazione/tabelle/tabella Richiede.png}
\end{figure}
\subsubsection{Tabella Categoria\_Corso}
\begin{figure}[h!]
    \centering 
    \includegraphics[width=0.8\textwidth]{CartellaDocumentazione/tabelle/tabella Categoria.png}
\end{figure}
\subsubsection{Tabella Specializzazione\_Chef}
\begin{figure}[h!]
    \centering 
    \includegraphics[width=0.8\textwidth]{CartellaDocumentazione/tabelle/tabella Specializzazione.png}
\end{figure}
\subsection{Creazione dei constraints}
\subsubsection{Su Corso}
\begin{figure}[h!]
    \centering 
    \includegraphics[width=0.8\textwidth]{CartellaDocumentazione/vincoli check/vincoli Corso.png}
\end{figure}
\subsubsection{Su Utente}
\begin{figure}[h!]
    \centering 
    \includegraphics[width=0.8\textwidth]{CartellaDocumentazione/vincoli check/vincoli Utente.png}
\end{figure}
\clearpage
\subsubsection{Su Ingrediente}
\begin{figure}[h!]
    \centering 
    \includegraphics[width=0.8\textwidth]{CartellaDocumentazione/vincoli check/vincoli Ingrediente.png}
\end{figure}
\subsubsection{Su Ricetta}
\begin{figure}[h!]
    \centering 
    \includegraphics[width=0.8\textwidth]{CartellaDocumentazione/vincoli check/vincoli Ricetta.png}
\end{figure}
\subsubsection{Su Sessione}
\begin{figure}[h!]
    \centering 
    \includegraphics[width=0.8\textwidth]{CartellaDocumentazione/vincoli check/vincoli Sessione.png}
\end{figure}
\subsubsection{Su Richiede}
\begin{figure}[h!]
    \centering 
    \includegraphics[width=0.8\textwidth]{CartellaDocumentazione/vincoli check/vincoli Richiede.png}
\end{figure}
\clearpage
\subsection{Triggers per il controllo dei vincoli}
\subsubsection{Su Adesione}
\begin{figure}[h!]
    \centering 
    \includegraphics[width=0.8\textwidth]{CartellaDocumentazione/vincoli trigger/adesione.png}
\end{figure}
\clearpage
\subsubsection{Su Iscrizione}
\begin{figure}[h!]
    \centering 
    \includegraphics[width=0.8\textwidth]{CartellaDocumentazione/vincoli trigger/iscrizione.png}
\end{figure}
\subsubsection{Su Compone}
\begin{figure}[h!]
    \centering 
    \includegraphics[width=0.8\textwidth]{CartellaDocumentazione/vincoli trigger/numero di sessioni.png}
\end{figure}
\clearpage
\begin{figure}[h!]
    \centering 
    \includegraphics[width=0.8\textwidth]{CartellaDocumentazione/vincoli trigger/modifica numero di sessioni.png}
\end{figure}

\subsection{Triggers per la gestione dei dati derivabili}
\subsubsection{Di num\_partecipanti}
\begin{figure}[h!]
    \centering 
    \includegraphics[width=0.8\textwidth]{CartellaDocumentazione/gestione derivabili/num Partecipanti.png}
\end{figure}
\clearpage
\subsubsection{Di num\_aderenti}
\begin{figure}[h!]
    \centering 
    \includegraphics[width=0.8\textwidth]{CartellaDocumentazione/gestione derivabili/num Aderenti.png}
\end{figure}
\subsection{Creazione viste}

\begin{figure}[h!]
    \centering 
    \includegraphics[width=0.8\textwidth]{CartellaDocumentazione/immagini/creazione view.png}
\end{figure}



\end{document}

