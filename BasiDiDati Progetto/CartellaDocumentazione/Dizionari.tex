\section{Dizionoari}
\subsection{Dizionario delle Classi}
\renewcommand{\arraystretch}{1.3}

\begin{table}[H] % oppure [htbp]
\caption{Dizionario delle Classi:1}
\label{tab:classi1}

\centering
\resizebox{0.9\textwidth}{!}{
\begin{tabular}{|>{\centering\arraybackslash}m{0.22\textwidth}|>
                {\centering\arraybackslash}m{0.28\textwidth}|>{\raggedright\arraybackslash}m{0.45\textwidth}|}

                \hline
    \textbf{Classe} & \textbf{Descrizione} & \textbf{Attributi} \\
    \hline
    \multirow{12}{=}{\centering Utente} & \multirow{12}{=}{\centering Descrittore di ogni Studente o Chef } &
\texttt{email} (String): è un indirizzo mail unico e valido che permette di contattare l’utente.\\

& & \texttt{nome} (String): è il nome dell' utente.\\

& & \texttt{cognome} (String): è cognome di ogni utente \\

& & \texttt{password} (String): una password permette ad ogni utente di proteggere il suo account e le sue informazioni.\\

& & \texttt{specializzazione} (String): specifica il titolo dello cehf\\

& & \texttt{matricola} (String): string alfanumerica che identifica lo studente \\ 

& & \texttt{tipoutente} (TipoUtente): indica il tipo di utente (chef) (studente) oppure (chefStudente) \\


\hline
\multirow{8}{=}{\centering Corso} & \multirow{8}{=}{\centering Descrittore di ciascun corso creato e partecipato dagli utenti.} &
\texttt{dataCorso} (Data): indica l'inizio del corso.\\

& & \texttt{nome} (String): è il nome che assume il corso. \\

& & \texttt{categoria} (String): sono tutte le categiore che può assumerre il corso \\

& & \texttt{frequenza} (Data): indica ogni quanti giorni il corso sì svloge\\

& & \texttt{numPartecipanti} (Int): Numero dei partecipanti facente parte dal corso \\
\hline

\multirow{3}{=}{\centering Iscrizione} & \multirow{3}{=}{\centering Descrittore di dell'iscrizione degli utenti.} &
\texttt{dataIscrizione} (Data): indica l'inizio dell'iscrizione per un corso.\\

& & \texttt{Stato} (String): indica lo stato di uno studente per l'iscrizione ad un corso \\

\hline

\end{tabular}
}

\end{table}




\begin{table}[H] % oppure [htbp]
\caption{Dizionario delle Classi: end}
\label{tab:classi2}
\centering
\resizebox{0.9\textwidth}{!}{
\begin{tabular}{|>{\centering\arraybackslash}m{0.22\textwidth}|>
                {\centering\arraybackslash}m{0.28\textwidth}|>{\raggedright\arraybackslash}m{0.45\textwidth}|}

                \hline
    \textbf{Classe} & \textbf{Descrizione} & \textbf{Attributi} \\
    \hline
    \multirow{12}{=}{\centering Sessione} & \multirow{12}{=}{\centering Descrittore di???} &
\texttt{numSessioni} (Int): specifica quante sessioni ha un corso in presenza. \\
& & \texttt{oraInizio} (Data): orario di inizio della sessione.\\

& & \texttt{numAderenti} (Int): quanti studenti partecipano alla sessione. \\

& & \texttt{quantità} (Double): la quantità degli ingredienti per la ogni studente cosi da evitare sprechi.\\

& & \texttt{teoria} (String): ??\\

& & \texttt{tipoSessione} (TipoSessione): Specifca del tipo di sessione se on-line o in presenza. \\ 



\hline
\multirow{6}{=}{\centering Ricetta} & \multirow{6}{=}{\centering Descrittore di??.} &
\texttt{nome} (String): nome con il quale si identifica la ricetta.\\

& & \texttt{descrizione} (String): è il procedimentio per comporre la ricetta. \\

& & \texttt{tempo} (Data): tempo necessario per fare una ricetta. \\


\hline

\multirow{4}{=}{\centering Ingrediente} & \multirow{4}{=}{\centering Descrittore di ??.} &
\texttt{nome} (String): nome dell'ingrediente per riconoscerlo.\\

& & \texttt{unitàDiMisura} (String): spechifica del tipo di misura di un ingrediente ES: Kg, litri ecc.. .\\

\hline


\centering Adesione & \centering Descrittore di ??. &
\texttt{dataAdesione} (Data): data relativa all'adesione di uno studente ad una sessione.\\


\hline

\centering Richiede & \centering Descrittore di ??. &
\texttt{quantitàNecessaria} (Double): calcola la quantità di ogni singolo ingrediente per una ricetta.\\

\hline

\end{tabular}
}

\end{table}



\subsection{Dizionoario Assocazioni}


\begin{table}[H] % oppure [htbp]
\caption{Dizionario delle assocazioni: }
\label{tab:classi} 
\centering
%\resizebox{1.\textwidth}{!}{
\begin{tabular}{|>{\centering\arraybackslash}m{0.22\textwidth}|>
                {\centering\arraybackslash}m{0.28\textwidth}|>{\raggedright\arraybackslash}m{0.45\textwidth}|}
               

                \hline
    \textbf{Associazine} & \textbf{Descrizione} & \textbf{Molteplicità delle Classi coinvolte} \\
    \hline
   \centering richiede & \centering Richiesta di Adesione da parte di uno studente & Utente [0..*] - Adesione [1] \\



\hline

  \centering richiede & \centering Richiesta di Iscrizione da parte di uno studente & Utente [0..*] - Iscrizione [1] \\

\hline

  \centering gestisce & \centering Gestione di un corso da parte di uno chef & Utente [1..*] - Corso [0..*] \\
\hline

  \centering compone & \centering Un corso è composto da vari tipi di sessioni  & Corso [1] - Sessione [1..*] \\
\hline  



  \centering prepara & \centering Ad ogni sesisone in presenza si prepara una ricetta & Sessione [0..*] - Ricetta [1..*] \\

\hline

  \centering haBisogno & \centering ogni ricetta haBisogno di sapere la quantità degli ingredienti  & Ricetta [1] - Richiede [1..*] \\

\hline


  \centering necessita & \centering Un ingrediente necesista di sapre la quantita per la ricetta & Richiede [1..*] - Ingrediente [1] \\

\hline

  \centering per & \centering Richiesta di adesione per una sessione in presenza & Adesione [1] - Sessione [0..*] \\

\hline

  \centering relativa & \centering Iscrizione di uno Studente per un corso & Inscrizione [1] - Corso [0..*] \\

\hline

\end{tabular}
%}

\end{table}

\clearpage
\subsection{Dizionario Vincoli}


\renewcommand{\arraystretch}{1.3}

\begin{center}
\begin{longtable}{|>{\centering\arraybackslash}m{0.30\textwidth}|
                  >{\centering\arraybackslash}m{0.22\textwidth}|
                  >{\raggedright\arraybackslash}m{0.45\textwidth}|}
\caption{Dizionario dei vincoli} \label{tab:vincoli} \\

\hline
\textbf{Vincolo} & \textbf{Tipo} & \textbf{Descrizione} \\
\hline
\endfirsthead

\hline
\textbf{Vincolo} & \textbf{Tipo} & \textbf{Descrizione} \\
\hline
\endhead

Email & Dominio & L’attributo “email” della classe Utente deve contenere una combinazione non nulla di lettere, numeri e simboli, seguiti da una chiocciola (“@”), altre lettere/simboli, un punto (“.”), e finire con almeno due lettere. \\
\hline

Password & Dominio & L’attributo “password” della classe Utente deve essere lungo almeno 8 caratteri, e avere almeno una lettera maiuscola, una minuscola, un numero e un carattere speciale. \\
\hline

Nome & Dominio & L’attributo “Nome” della classe Utente deve contenere solo stringhe di caratteri alfabetici, non vuote e con lunghezza <= 50 caratteri. \\
\hline

Cognome & Dominio & L’attributo “Cognome” della classe Utente deve contenere solo stringhe di caratteri alfabetici, non vuote e con lunghezza <= 50 caratteri. \\
\hline

TipoUtente & Dominio & L’attributo "tipoUtente" della classe Utente è un’enumerazione che può assumere solo i valori: chef, studente, chefStudente, per identificare il ruolo dell’utente nel sistema. \\
\hline

Specializzazione & N‑tupla & Se l’attributo “TipoUtente” è chef o chefStudente, allora l’attributo “Specializzazione” assume un valore alfabetico con lunghezza <= 50. \\
\hline

Matricola & N‑tupla & Se l’attributo “TipoUtente” è studente o chefStudente, allora l’attributo “Matricola” assume un valore alfanumerico con lunghezza <= 50. \\
\hline

DataInizio & Dominio & L'attributo "DataInizio" della classe Corso deve essere una data valida nel formato previsto, non può essere nulla. \\
\hline

Nome & Dominio & L'attributo "Nome" della classe Corso deve contenere una stringa non vuota con solo caratteri alfabetici e lunghezza massima 50 caratteri. \\
\hline

Categoria & N‑tupla & L'attributo "Categoria" della classe Corso deve contenere almeno una categoria. Ogni elemento deve essere una stringa non vuota. \\
\hline

Frequenza & Dominio & L'attributo "Frequenza" della classe Corso deve contenere una stringa che rappresenta un intervallo temporale coerente. \\
\hline

NumPartecipanti & Intrarelazionale & L'attributo "NumPartecipanti" della classe Corso è calcolato automaticamente in base al numero di iscrizioni associate. \\
\hline

NumSessioni & Dominio & L'attributo "NumSessioni" della classe Sessione deve essere un intero positivo >0. \\
\hline

OraInizio & Dominio & L'attributo "OraInizio" della classe Sessione deve essere una data valida, non nulla. \\
\hline

TipoSessione & Dominio & L'attributo "TipoSessione" della classe Sessione è un’enumerazione che può assumere solo i valori: presenza, online. \\
\hline

Quantità & Intrarelazionale & Se l'attributo "TipoSessione" è "presenza", allora "Quantità" assume un valore numerico >0. \\
\hline

NumAderenti & Intrarelazionale & Se l'attributo "TipoSessione" è "presenza", allora "NumAderenti" assume un valore numerico valido >0. \\
\hline

Teoria & Intrarelazionale & Se l'attributo "TipoSessione" è "online", allora "Teoria" deve essere una stringa alfanumerica con lunghezza <=255. \\
\hline

Nome & Dominio & L'attributo "Nome" della classe Ricetta deve contenere una stringa alfabetica non vuota con lunghezza <=50. \\
\hline

Descrizione & Dominio & L'attributo "Descrizione" della classe Ricetta deve contenere una stringa alfabetica non vuota con lunghezza <=50. \\
\hline

Tempo & Dominio & L'attributo "Tempo" della classe Ricetta deve contenere una data (ora, minuti, secondi) valida e non vuota. \\
\hline

Nome & Dominio & L'attributo "Nome" della classe Ingrediente deve contenere una stringa alfabetica non vuota con lunghezza <=50. \\
\hline

UnitàDiMisura & Dominio & L'attributo "UnitàDiMisura" della classe Ingrediente deve contenere una stringa alfabetica non vuota con lunghezza <=50. \\
\hline

\end{longtable}
\end{center}

