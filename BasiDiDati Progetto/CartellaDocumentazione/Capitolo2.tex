\section{Progettazione Concettuale}
\subsection{Diagramma ER}

\begin{figure}[!h] % [h] = here, posizione suggerita
    \centering
    \includegraphics[width=0.8\textwidth]{CartellaDocumentazione/DiagrammaER1.pdf}
\end{figure}

\subsubsection{Descrizione Diagramma ER}
Il diagramma descrive il database per la piattaforma UninFoodLab, includendo la gestione degli utenti, la pianificazione delle sessioni e la logistica degli ingredienti.
\subsubsection{Gestione Utenti e Gerarchie}
L'entità padre è Utente, che raccoglie i dati comuni: \textit{Nome},\textit{ Cognome}, \textit{Email} e \textit{Password}. Esiste una generalizzazione che divide l'utente in due ruoli. La doppia linea sotto "Utente" suggerisce una generalizzazione totale, mentre i cerchi con la lettera "O" indicano una specializzazione overlapping:
\begin{itemize}
    \item \textbf{Studente}: Identificato da una Matricola;
    \item \textbf{Chef}: Caratterizzato da una Specializzazione.
\end{itemize}
\subsubsection{Corsi e Iscrizioni}
L'entità centrale dell'offerta formativa è il Corso.
\begin{itemize}
    \item Relazione \textbf{Gestisce}: Collega Chef e Corso.
    \begin{itemize}
        \item Uno Chef gestisce da 0 a N corsi;
        \item  Un corso deve essere gestito da almeno 1 chef per esistere (cardinalità 1,N)f.
    \end{itemize}
    \item Relazione \textbf{Iscrizione}: Collega Studente e Corso.
    \begin{itemize}
        \item È una relazione molti-a-molti con attributi propri: \textit{DataIscrizione} e \textit{Stato};
        \item Uno studente può avere 0 a N iscrizioni;
        \item Un corso deve avere almeno 1 iscritto per esistere (cardinalità 1,N).
    \end{itemize}
\end{itemize}
\subsubsection{Struttura delle Sessioni}
Il corso è strutturato in lezioni tramite la relazione \textbf{Compone}.
\begin{itemize}
    \item Cardianalità: Un Corso è composto da 1 a N sessioni , mentre una Sessione appartiene a 1 solo corso;
    \item Entità Sessione : Ha un attributo OraInizio;
    \item Gerarchia Sessioni: È presente una generalizzazione Disgiunta (indicata dalla lettera "D" ), il che significa che una sessione è o online o in presenza, mai entrambe:
    \begin{itemize}
        \item SessioneOnline: Ha l'attributo specifico \textit{Teoria}.
        \item SessionePresenza: È l'entità che gestisce la logistica fisica.
    \end{itemize}
\end{itemize}
\subsubsection{Logistica in Presenza (Adesioni e Ricette)}
Questa parte è specifica per le SessionePresenza:
\begin{itemize}
    \item Adesione Studenti: Gli studenti partecipano fisicamente tramite la relazione \textbf{Adesione}, che ha l'attributo \textit{DataAdesione};
    \begin{itemize}
        \item La SessionePresenza ha un attributo derivato \textit{NumeroAderenti} per contare i presenti;
    \end{itemize}
    \item Preparazione Ricette: La relazione \textbf{Prepara} collega la sessione alle ricette;
    \begin{itemize}
        \item Ha un attributo derivato importante: \textit{QuantitàTotale};
        \item Cardinalità: Una sessione prepara 1 a N ricette , e una ricetta può essere preparata in 1 a N sessioni.
    \end{itemize}
\end{itemize}
\clearpage
\subsection{Diagramma UML}
\begin{figure}[!h] % [h] = here, posizione suggerita
    \centering
   \includegraphics[width=1.0\textwidth]{CartellaDocumentazione/DiagrammaUML1.pdf}
\end{figure}
\subsubsection{Descrizione Diagramma UML}
\subsubsection{Tipi di Dato e Precisione Numerica}
Il diagramma UML definisce esattamente come i dati devono essere memorizzati, sciogliendo le ambiguità generiche dell'ER:
\begin{itemize}
\item Gestione delle Quantità: Si nota una distinzione importante tra Int e Double;
\item Contatori come \textit{numPartecipanti} e \textit{numAderenti} sono Int, poiché contano unità discrete (persone, lezioni);
\item Le misure fisiche come \textbf{quantitàTotale} e \textbf{quantitàNecessaria}  sono definite come Double. Questo indica che il sistema gestirà valori decimali precisi (es. 1.5 kg, 0.25 litri), dettaglio fondamentale per le ricette che l'ER lasciava implicito.
\item Date e Orari: i riferimenti temporali sono distinti in base al significato: \textit{dataAdesione} e \textit{dataIscrizione} sono Date, \textit{oraInizio} è un DateTime (data+ora), mentre \textit{tempo} è un Time (ora, minuti, secondi).
\end{itemize}
\subsubsection{Rappresentazione degli Array (Attributi Multivalore)}
L'UML traduce graficamente gli attributi multivalore dell'ER con la notazione delle parentesi quadre:
\begin{itemize}
\item +specializzazione[1...*]: String nella classe Chef;
\item +categoria[1...*]: String nella classe Corso. 
\end{itemize}
Questa notazione dice esplicitamente che posso avere più valori.
\clearpage

\subsection{Ristrutturazione Diagramma UML}
\begin{figure}[!h] % [h] = here, posizione suggerita
    \centering
    \includegraphics[width=1.0\textwidth]{CartellaDocumentazione/DiagrammaUMLRistrutt1.pdf}
\end{figure}

\subsubsection{Descrizione Delle Scelte}
Nel modello iniziale era presente una generalizzazione della classe Utente nelle sottoclassi Studente e Chef. Abbiamo deciso di adottare la seconda strategia, accorpando le due sottoclassi all’interno della classe padre Utente. Questa scelta è motivata dall’assenza di attributi propri nella classe Studente, che avrebbe reso la distinzione strutturale poco significativa. Per mantenere comunque la distinzione dei ruoli, è stato introdotto un attributo discriminante \textit{TipoUtente}.

Analogamente, nel modello iniziale era presente una generalizzazione della classe Sessione nelle sottoclassi SessionePresenza e SessioneOnline. Anche in questo caso è stata adottata la strategia di accorpamento nella classe padre, in quanto la classe SessioneOnline offriva un contributo informativo limitato e non tale da giustificare una modellazione separata. È stato pertanto introdotto un attributo discriminante \textit{TipoSessione}. Questo attributo consente di gestire correttamente la natura disgiunta della generalizzazione e di distinguere in modo chiaro le diverse modalità di erogazione delle sessioni.
\subsubsection{Accorpamento della gerarchia Utente}
Invece di avere tre classi distinte (Utente, Chef, Studente), ora esiste una sola classe Utente:
\begin{itemize}
    \item Enumerazione come discriminatore: È stata introdotta l'enumerazione TipoUtente che contiene i valori \textbf{chef}, \textbf{studente} e il valore combinato \textbf{chefStudente}. Questo gestisce esplicitamente il caso di Overlapping;
    \item Fusione degli attributi: La classe Utente contiene ora tutti gli attributi possibili di tutti i ruoli, ma con cardinalità facoltativa:
    \begin{itemize}
        \item +\textit{matricola}[0...1]: String: Sarà valorizzata solo se l'utente è uno studente.
        \item +\textit{specializzazione}[0...*]: String: Sarà valorizzata solo se l'utente è uno chef.
    \end{itemize}
\end{itemize}
\subsubsection{Accorpamento della gerarchia Sessione }
Analogamente a Utente, la distinzione tra sessioni online e in presenza è stata rimossa strutturalmente per creare un'unica classe Sessione:
\begin{itemize}
    \item Enumerazione come discriminatore: È introdotta l'enumerazione TipoSessione che contiene i valori \textit{Presenza} e \textit{Online} per distinguere la natura della lezione.
    \item Fusione degli attributi: La classe Sessione contiene ora tutti gli attributi che prima erano separati, rendendoli però opzionali:
    \begin{itemize}
        \item +\textit{teoria}[0...1]: String: Usato solo se il tipo è online;
        \item\textit{<<derivate>>numAderenti}[0...1]: Int e \textit{<<derivate>>+quantità}[0...1]: double: Usati solo se il tipo è presenza.
    \end{itemize}
\end{itemize}