

\renewcommand{\arraystretch}{1.3}

\begin{center}
\begin{longtable}{|>{\centering\arraybackslash}m{0.30\textwidth}|
                  >{\centering\arraybackslash}m{0.22\textwidth}|
                  >{\raggedright\arraybackslash}m{0.45\textwidth}|}

\hline
\textbf{Vincolo} & \textbf{Tipo} & \textbf{Descrizione} \\
\hline
\endfirsthead

\hline
\textbf{Vincolo} & \textbf{Tipo} & \textbf{Descrizione} \\

\endhead
email .. PRIMARY KEY & Integrità dell'entità & Identifica univocamente ogni utente. Garantisce l'unicità dell'email e che essa non sia nulla.\\
\hline

check\_email & Dominio & L’attributo “email” della classe Utente deve contenere una combinazione non nulla di lettere, numeri e simboli, seguiti da una chiocciola (“@”), altre lettere/simboli, un punto (“.”), e finire con almeno due lettere. \\
\hline
password VARCHAR(255) NOT NULL & Dominio & Impone che l'attributo "password" della classe Utente sia una stringa alfanumerica <= 255 caratteri e che sia obbligatoriamente non nullo.\\
\hline
check\_password & Dominio & L’attributo “password” della classe Utente deve essere lungo almeno 8 caratteri, e avere almeno una lettera maiuscola, una minuscola, un numero e un carattere speciale. \\
\hline
nome VARCHAR(50) NOT NULL (Utente) & Dominio & Impone che l'attributo "nome" della classe Utente sia una stringa alfanumerica <= 50 caratteri e che sia obbligatoriamente non nullo.\\
\hline
check\_nome\_utente & Dominio & L’attributo “nome” della classe Utente deve contenere solo stringhe di caratteri alfabetici. \\
\hline
cognome VARCHAR(50) NOT NULL & Dominio & Impone che l'attributo "cognome" della classe Utente sia una stringa alfanumerica <= 50 caratteri e che sia obbligatoriamente non nullo.\\
\hline
check\_cognome & Dominio & L’attributo “cognome” della classe Utente deve contenere solo stringhe di caratteri alfabetici. \\
\hline

tipo\_utente tipo\_utente\_enum NOT NULL & Dominio & L’attributo "tipoUtente" della classe Utente è un’enumerazione che può assumere solo i valori: chef, studente, chefStudente. \\
\hline

check\_ruolo\_dati & Tupla & Se l’attributo “tipoUtente” della classe Utente è studente o chefStudente, allora l’attributo “matricola” assume un valore alfanumerico <= 20. \\
\hline
matricola .. UNIQUE & Intrarelazionale & Impone l'assenza di duplicati per l'attributo, assicurando che ad ogni matricola corrisponda un solo profilo registrato nel sistema.\\
\hline
data\_inizio DATE NOT NULL & Dominio & L'attributo "dataInizio" della classe Corso deve essere una data valida nel formato previsto, non può essere nulla. \\
\hline
nome VARCHAR(50) NOT NULL (Corso) & Dominio & Impone che l'attributo "nome" della classe Corso sia una stringa alfanumerica <= 50 caratteri e che sia obbligatoriamente non nullo.\\
\hline
check\_nome\_corso & Dominio & L'attributo "nome" della classe Corso deve contenere una stringa  con solo caratteri alfabetici. \\
\hline
frequenza VARCHAR(50) NOT NULL & Dominio & Impone che l'attributo "frequenza" ella classe Corso sia una stringa alfanumerica <= 50 caratteri e che sia obbligatoriamente non nullo. \\
\hline
trg\_partecipanti & Interrelazionale & L'attributo "numPartecipanti" della classe Corso è calcolato automaticamente in base al numero di iscrizioni associate. \\
\hline
num\_sessioni .. NOT NULL & Dominio & Impone che l'attributo "numSessioni" della classe Corso sia obbligatoriamente non nullo.\\
\hline
check\_num\_sessioni & Dominio & L'attributo "numSessioni" della classe Corso deve essere un intero positivo >0. \\
\hline
 ora\_inizio .. NOT NULL & Dominio & L'attributo "oraInizio" della classe Sessione deve essere una data valida nel formato previsto, non può essere nulla. \\
\hline
tipo\_sessione tipo\_sessione\_enum NOT NULL & Dominio & L'attributo "tipoSessione" della classe Sessione è un’enumerazione che può assumere solo i valori: presenza o online. \\
\hline
check\_logica\_sessione & Tupla & Se l'attributo "tipoSessione" della classe Sessione è "presenza", allora "teoria" deve essere NULL.  Se l'attributo "tipoSessione" è "online", allora "teoria" deve essere una stringa alfanumerica con lunghezza <=255. \\
\hline

trg\_aderenti & Interrelazionale & Se l'attributo "tipoSessione" della classe Sessione è "presenza", allora "numAderenti"  è calcolato automaticamente in base al numero di adesioni associate.\\
\hline

trg\_check\_iscrizione & Interrelazionale & La relazione "Iscrizione" può coinvolgere solo utenti con tipoUtente = studente o chefStudente. \\
\hline

trg\_check\_adesione & Interrelazionale & La relazione "Adesione" può coinvolgere solo utenti con tipoUtente = studente o chefStudente e solo sessioni con tipoSessione = presenza. Inoltre verifica che l'utente sia iscritto al corso. \\
\hline

trg\_check\_prepara & Interrelazionale & La relazione "Prepara" può coinvolgere solo sessioni con tipoSessione = presenza. Ogni sessione in presenza prepara almeno una ricetta e ogni ricetta è preparata in almeno una sessione. \\
\hline
nome VARCHAR(50) NOT NULL (Ricetta) & Dominio & Impone che l'attributo "nome" della classe Ricetta sia una stringa alfanumerica <= 50 caratteri e che sia obbligatoriamente non nullo.\\
\hline
check\_nome\_ricetta & Dominio & L'attributo "nome" della classe Ricetta deve contenere una stringa alfabetica. \\
\hline
descrizione VARCHAR(250) NOT NULL & Dominio & Impone che l'attributo "descrizione" della classe Ricetta sia una stringa alfanumerica <= 250 caratteri e che sia obbligatoriamente non nullo.\\
\hline
check\_descrizione & Dominio & L'attributo "descrizione" della classe Ricetta deve contenere una stringa alfabetica. \\
\hline
tempo TIME NOT NULL & Dominio & L'attributo "tempo" della classe Ricetta deve contenere un orario (ore, minuti, secondi) valido e non nullo. \\
\hline
nome .. PRIMARY KEY (Ingrediente) & Integrità dell'entità & Identifica univocamente ogni ingrediente. Garantisce l'unicità del nome e che esso non sia nullo.\\
\hline
check\_nome\_ingrediente & Dominio & L'attributo "nome" della classe Ingrediente deve contenere una stringa alfabetica. \\
\hline
unita\_di\_misura VARCHAR(20) NOT NULL & Dominio &  Impone che l'attributo "unitàDiMisura" della classe Ingrediente  sia una stringa alfanumerica <= 20 caratteri e che sia obbligatoriamente non nullo.\\
\hline
check\_unita\_di\_misura & Dominio & L'attributo "unitàDiMisura" della classe Ingrediente deve contenere una stringa alfabetica \\
\hline
trg\_check\_ specializzazione & Interrelazionale & La relazione "specializzatoIn" può coinvolgere solo utenti con tipoUtente uguale a chef o chefStudente. Un utente con ruolo di solo Studente non può possedere specializzazioni. \\
\hline
trg\_check\_gestisce & Interrelazionale & La relazione "Gestisce" può coinvolgere solo utenti con tipoUtente uguale a chef o chefStudente. Un utente con ruolo di solo Studente non può gestire Corsi o Sessioni.\\
\hline
trg\_check\_max\_ sessioni\_per\_corso & Interrelazionale & Impedisce l'inserimento o la modifica di tuple nella tabella Sessione qualora ciò comporti il superamento del valore "numSessioni" definito nella tabella Corso associata.\\
\hline
trq\_check\_num\_ sessioni\_corso\_update & Interrelazionale & Inibisce l'aggiornamento del attributo "numSessioni" nella tabella Corso verso un valore inferiore alla cardinalità attuale delle tuple correlate nella tabella Sessione, preservando la consistenza dei dati esistenti.\\
\hline


\end{longtable}
\end{center}
