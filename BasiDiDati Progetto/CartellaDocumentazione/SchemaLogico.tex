
\subsection{Schema Logico}
Prima di procedere all'implementazione fisica del database, è necessario definire lo schema logico. Tale modello si ottiene traducendo le entità e le associazioni dello schema concettuale in tabelle vere e proprie.\\ Di seguito la notazione utilizzata per la lettura:
\begin{itemize}
\item Sottolineatura singola per le chiavi primarie;
\item Doppia sottolineatura per le chiavi esterne;
\item Carattere corsivo per le parti costituenti una chiave primaria composta.
\end{itemize}
\subsubsection{Traduzione delle classi}
Dalla trasformazione del diagramma delle classi deriva la seguente organizzazione delle tabelle relazionali:\\
    \vspace{0.2cm}
    {\bfseries Utente} (\underline {email}, nome, cognome, password, matricola, tipoUtente)\\
    \hrule 
    \vspace{0.2cm}
    {\bfseries Corso} (\underline {id\_corso}, dataInizio, nome, frequenza, numPartecipanti,   
    \\numSessioni)\\
    
    \hrule
    \vspace{0.2cm}
    {\bfseries Sessione} (\underline {id\_sessione}, oraInizio, numAderenti, teoria, \\TipoSessione, \underline{\underline{id\_corso}})\\
    \colorbox{yellow}{Chiavi esterne: id\_corso $\rightarrow$ Corso.id\_corso}\\
    
    \hrule
    \vspace{0.2cm}
    {\bfseries Ricetta} (\underline {id\_ricetta}, nome, descrizione, tempo)\\
    
    \hrule
    \vspace{0.2cm}
    {\bfseries Ingrediente} (\underline{nome}, unitàDiMisura)\\
    
    \hrule
    \vspace{0.2cm}
    {\bfseries Adesione} (dataAdesione, \textit{email, id\_sessione}, \underline{\underline{email}}, \underline{\underline{id\_sessione}})\\
    \colorbox{yellow}{%
    \begin{minipage}{0.7\textwidth}
    Chiavi esterne: id\_sessione $\rightarrow$ Sessione.id\_sessione \\
     \space email $\rightarrow$ Utente.email
  \end{minipage}%
   }\\
   
   \hrule
   \vspace{0.2cm}
    {\bfseries Iscrizione} (dataIscrizione, stato, \textit{email, id\_corso}, \underline{\underline{email}}, \underline{\underline{id\_corso}})\\
    \colorbox{yellow}{%
    \begin{minipage}{0.7\textwidth}
    Chiavi esterne: id\_corso $\rightarrow$ Corso.id\_corso \\
     \space email $\rightarrow$ Utente.email
  \end{minipage}%
   }\\
   
   \hrule
   \vspace{0.2cm}
    {\bfseries Richiede} (\textit{id\_ricetta}, \textit{nomeIngrediente}, \underline{\underline{id\_ricetta}}, \underline{\underline{nomeIngrediente}}, \\quantitàNecessaria)\\
    \colorbox{yellow}{%
    \begin{minipage}{0.7\textwidth}
    Chiavi esterne: id\_ricetta $\rightarrow$ Ricetta.id\_ricetta \\
     \space nomeIngrediente $\rightarrow$ Ingrediente.nome
  \end{minipage}%
   }\\
   
   \hrule
   \vspace{0.2cm}
    {\bfseries Specializzazione\_Chef} (\textit{email\_chef, specializzazione}, \underline{\underline{email\_chef}})\\
    \colorbox{yellow}{Chiave esterna: email\_chef $\rightarrow$ Utente.email}\\
    
    \hrule
    \vspace{0.2cm}
    {\bfseries Categoria\_Corso} (\textit{id\_corso, categoria}, \underline{\underline{id\_corso}})\\
     \colorbox{yellow}{Chiave esterna: id\_corso $\rightarrow$ Corso.id\_corso}\\
     
     \hrule
     \vspace{0.2cm}
    {\bfseries Gestisce} (\textit{email\_chef, id\_corso}, \underline{\underline{email\_chef}}, \underline{\underline{id\_corso}})\\
    \colorbox{yellow}{%
    \begin{minipage}{0.6\textwidth}
    Chiavi esterne: id\_corso $\rightarrow$ Corso.id\_corso \\
     \space email\_chef $\rightarrow$ Utente.email
  \end{minipage}%
   }\\
   
   \hrule
   \vspace{0.2cm}
    {\bfseries Prepara} (\textit{id\_sessione, id\_ricetta}, \underline{\underline{id\_sessione}}, \underline{\underline{id\_ricetta}})\\
    \colorbox{yellow}{%
    \begin{minipage}{0.7\textwidth}
    Chiavi esterne: id\_ricetta $\rightarrow$ Ricetta.id\_ricetta \\
     \space id\_sessione $\rightarrow$ Sessione.id\_sessione
  \end{minipage}%
   }\\
   \hrule
   \vspace{0.2cm}
    {\bfseries Vista\_Fabbisogni\_Sessione} (\textit{id\_sessione, nomeIngrediente}, \underline{\underline{id\_sessione}}, \underline{\underline{nomengrediente}}, quantitàTotale)\\
    \colorbox{yellow}{%
    \begin{minipage}{0.7\textwidth}
    Chiavi esterne: nomeIngrediente $\rightarrow$ Ingrediente.nome \\
     \space id\_sessione $\rightarrow$ Sessione.id\_sessione
  \end{minipage}%
   }\\
\subsubsection{Traduzione delle associazioni}

\begin{table}[H]
\centering
\renewcommand{\arraystretch}{1.2}

\begin{tabular}{|p{0.30\textwidth}|p{0.65\textwidth}|}
\hline
\textbf{Associazione} & \textbf{Implementazione} \\
\hline

\textbf{adesione} \newline \textit{(Utente - Adesione)} &
\textbf{Importazione Chiave Esterna:} L'entità \textit{Adesione} (lato N) importa la chiave primaria di \textit{Utente} (\texttt{id\_utente}) come chiave esterna. \\
\hline

\textbf{iscrizione} \newline \textit{(Utente - Iscrizione)} &
\textbf{Importazione Chiave Esterna:} L'entità \textit{Iscrizione} importa la chiave primaria di \textit{Utente} (\texttt{id\_utente}) come chiave esterna. \\
\hline

\textbf{gestisce} \newline \textit{(Chef - Corso)} &
\textbf{Importazione Chiave Esterna:} Data la cardinalità 1:N, la tabella \textit{Corso} ospita la chiave esterna dello \textit{Chef} (\texttt{id\_chef}). \\
\hline

\textbf{compone} \newline \textit{(Corso - Sessione)} &
\textbf{Importazione Chiave Esterna:} La tabella \textit{Sessione} importa la chiave primaria di \textit{Corso} (\texttt{id\_corso}) come chiave esterna. \\
\hline

\textbf{prepara} \newline \textit{(Sessione - Ricetta)} &
\textbf{Nuova Tabella (Ponte):} Data la cardinalità Molti-a-Molti, si crea la tabella di raccordo \textit{Prepara} composta dalle chiavi esterne delle due entità (\texttt{id\_sessione}, \texttt{id\_ricetta}). \\
\hline

\textbf{haBisogno} \newline \textit{(Ricetta - Richiede)} &
\textbf{Importazione Chiave Esterna:} La tabella \textit{Richiede} importa la chiave primaria della \textit{Ricetta} (\texttt{id\_ricetta}). \\
\hline

\textbf{necessita} \newline \textit{(Richiede - Ingrediente)} &
\textbf{Importazione Chiave Esterna:} La tabella \textit{Richiede} importa la chiave primaria dell'\textit{Ingrediente} (\texttt{id\_ingrediente}). \\
\hline

\textbf{per} \newline \textit{(Adesione - Sessione)} &
\textbf{Importazione Chiave Esterna:} La tabella \textit{Adesione} importa la chiave primaria della \textit{Sessione} (\texttt{id\_sessione}). \\
\hline

\textbf{relativa} \newline \textit{(Iscrizione - Corso)} &
\textbf{Importazione Chiave Esterna:} La tabella \textit{Iscrizione} importa la chiave primaria del \textit{Corso} (\texttt{id\_corso}). \\
\hline

\textbf{specializzatoIn} \newline \textit{(Chef - Specializzazione)} &
\textbf{Nuova Tabella:} La tabella \textit{Specializzazione\_Chef} viene creata per gestire l'attributo multivalore, importando la chiave primaria dello chef (\texttt{email\_chef}). \\
\hline

\textbf{categorizzatoIn} \newline \textit{(Corso - Categoria)} &
\textbf{Nuova Tabella:} La tabella \textit{Categoria\_Corso} viene creata per gestire l'attributo multivalore, importando la chiave primaria del corso (\texttt{id\_corso}). \\
\hline

\end{tabular}
\end{table}
   \clearpage
\subsubsection{Schema logico finale}
Di seguito viene riportato lo schema logico finale risultante, completo di chiavi primarie, esterne e attributi:\\

\renewcommand{\arraystretch}{1.80}
\begin{tabular}{lp{0.7\textwidth}} \hline
    {\bfseries Utente} & \underline {email}, nome, cognome, password, matricola, tipoUtente\\
    \hline
    {\bfseries Corso} & \underline {id\_corso}, dataInizio, nome, frequenza, numPartecipanti, numSessioni\\
    \hline
    {\bfseries Sessione} & \underline {id\_sessione}, oraInizio, numAderenti, teoria, TipoSessione, \underline{\underline{id\_corso}}\\
    \hline
    {\bfseries Ricetta} & \underline {id\_ricetta}, nome, descrizione, tempo\\
    \hline
    {\bfseries Ingrediente} & \underline {nome}, unitàDiMisura\\
    \hline
    {\bfseries Adesione} & dataAdesione, \textit{email, id\_sessione}, \underline{\underline{email}}, \underline{\underline{id\_sessione}}\\
    \hline
    {\bfseries Iscrizione} & dataIscrizione, stato, \textit{email, id\_corso}, \underline{\underline{email}}, \underline{\underline{id\_corso}}\\
    \hline
    {\bfseries Richiede} & quantitàNecessaria, \textit{id\_ricetta, nomeIngrediente}, \underline{\underline{id\_ricetta}}, \underline{\underline{nomeIngrediente}}\\
    \hline
    {\bfseries Specializzazione\_Chef} & \textit{email\_chef, specializzazione}, \underline{\underline{email\_chef}}\\
    \hline
    {\bfseries Categoria\_Corso} & \textit{id\_corso, categoria}, \underline{\underline{id\_corso}}\\
    \hline
    {\bfseries Gestisce} & \textit{email\_chef, id\_corso}, \underline{\underline{email\_chef}}, \underline{\underline{id\_corso}}\\
    \hline
    {\bfseries Prepara} & \textit{id\_sessione, id\_ricetta}, \underline{\underline{id\_sessione}}, \underline{\underline{id\_ricetta}}\\
    \hline
    {\bfseries Vista\_Fabbisogni\_Sessione} &\textit{id\_sessione}, \textit{nomeIngrediente}, \underline{\underline{id\_sessione}}, \underline{\underline{nomengrediente}}, quantitàTotale\\
    \hline
\end{tabular}

