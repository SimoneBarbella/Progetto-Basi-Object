\section{Schema Logico}
\subsection{Tabelle}
Leggenda:
\begin{itemize}
    \item Le parole con il {\bfseries grassetto} sono indicate tutti i nomi delle tabelle
    \item Le parole \underline {sottolineate} una volta indicano le chiavi primarie
    \item Le parole \underline{\underline{sottolinate}} due volte indicano le chiavi esterne
    \item Le parole in \textit{italico} indicano che una chiava primaria è composta da piu attributi 
\end{itemize}

\vspace{2em}

\renewcommand{\arraystretch}{1.80}
\begin{tabular}{lp{0.7\textwidth}} \hline
    {\bfseries Utente} & \underline {email}, nome, cognome, password, matricola, tipoUtente\\
    \hline
    {\bfseries Corso} & \underline {id\_corso}, dataInizio, nome, frequenza, numPartecipanti, numSessioni\\
    \hline
    {\bfseries Sessione} & \underline {id\_sessione}, oraInizio, numAderenti, quantità, teoria, TipoSessione, \underline{\underline{id\_corso}}\\
    \hline
    {\bfseries Ricetta} & \underline {id\_ricetta}, nome, descrizione, tempo\\
    \hline
    {\bfseries Ingrediente} & \underline {nome}, unitàDiMisura\\
    \hline
    {\bfseries Adesione} & dataAdesione, \textit{email, id\_sessione}, \underline{\underline{email}}, \underline{\underline{id\_sessione}}\\
    \hline
    {\bfseries Iscrizione} & dataIscrizione, stato, \textit{email, id\_corso}, \underline{\underline{email}}, \underline{\underline{id\_corso}}\\
    \hline
    {\bfseries Richiede} & quantitàNecessaria, \textit{id\_ricetta, nomeIngrediente}, \underline{\underline{id\_ricetta}}, \underline{\underline{nomeIngrediente}}\\
    \hline
    {\bfseries Specializzazione\_Chef} & \textit{email\_chef, specializzazione}, \underline{\underline{email\_chef}}\\
    \hline
    {\bfseries Categoria\_Corso} & \textit{id\_corso, categoria}, \underline{\underline{id\_corso}}\\
    \hline
    {\bfseries Gestisce} & \textit{email\_chef, id\_corso}, \underline{\underline{email\_chef}}, \underline{\underline{id\_corso}}\\
    \hline
    {\bfseries Prepara} & \textit{id\_sessione, id\_ricetta}, \underline{\underline{id\_sessione}}, \underline{\underline{id\_ricetta}}\\
    \hline
\end{tabular}
