\section{Schema Logico}
\subsection{Tabelle}
Leggenda:
\begin{itemize}
    \item Le parole con il \textbf{grassetto} sono indicate tutti i nomi delle tabelle
    \item Le parole \underline {sottolinete} una volta indicano le chiavi primarie
    \item Le parole \underline{\underline{sottolinate}} due volte indicano le chiavi esterne
    \item Le parole in \textit{italico} indicano che una chiava primaria è composta da piu attributi 
\end{itemize}

\vspace{2em}

\renewcommand{\arraystretch}{1.80}
\begin{tabular}{lp{0.7\textwidth}} \hline
    \textbf{Utente} & email, nome, cognome, password, specializzazione, matricola, tipoUtente, id\_corso\\ 
    \hline
    \textbf{Corso} & id\_corso, dataInizio, nome, categoria, frequenza, numPartecipanti, id\_sessione \\
    \hline
    \textbf{Sessione} & id\_sessione, numSessioni, oraInizio, numAderenti, quantità, teoria, TipoSessione, id\_ricetta, id\_corso\\
    \hline
    \textbf{Ricetta} & id\_ricetta, nome, descrizione, tempo, id\_sessione \\
    \hline
    \textbf{Ingrediente} & nome, unitàDiMisura\\
    \hline
    \textbf{Adesione} & dataAdesione, email,id\_sessione\\
    \hline
    \textbf{Iscrizione} & dataIscrizione, stato, email,id\_corso\\
    \hline
    \textbf{Richiede} & quantitàNecessaria, id\_ricetta, nomeIngrediente \\
    \hline
\end{tabular}
